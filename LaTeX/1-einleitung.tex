\section{Datenerfassung auf Intensivstationen}

\lipsum[1-3]
\section{Maschinelles Lernen}
\lipsum[4]
\begin{quote}
    {\foreignlanguage{english}{A computer program is said to learn from experience E with respect to some class of tasks T and performance measure P, if its performance at tasks in T, as measured by P, improves with experience E.}}
\end{quote}
\lipsum[5]
\subsection{Klassifikation vs Regression}
siehe: https://stats.stackexchange.com/a/282890 Handelt es sich um \glqq ordinal classification\grqq{}?

Eigentlich ist es eine Einordnung der Daten in diskrete Klassen/Scores. Aber Argumente für Regression sind:
\begin{itemize}
    \item Betrachtet man die Scores als separate Klassen geht verloren, dass z.b. 2 näher an 3 ist als an -4.
    \item insbesondere bei gängigen performance-metriken wie accuracy gibt es nur "richtig" oder falsch.
    \item floats können zu integer gerundet werden, anhand der abweichung könnte man sogar eine \glqq confidence\grqq{} ermitteln.
\end{itemize}