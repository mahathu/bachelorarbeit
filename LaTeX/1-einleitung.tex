\section{Datenerfassung auf Intensivstationen}
%   * worum geht es?
%   * Thematik
%     * Problematik der Datenerfassung auf ICUs

\cite{marxIntensivmedizin2015c} verorten die Intensivmedizin im Spannungsfeld zwischen Heilen und Sterben. Eine Verlegung auf die Intensivstation erfolgt häufig infolge einer besonders schweren oder lebensbedrohlichen Erkrankung oder Verletzung. 
Ziel der Behandlung ist es, sofern möglich, den Patienten\footnote{Hier und im Rest der Arbeit umfasst das generische Maskulinum, sofern nicht anders angegeben, Personen beider Geschlechter.} insofern zu kurieren, dass diesem ein Weiterleben unabhängig von den besonderen technischen und personellen Möglichkeiten der Intensivmedizin möglich ist. Dafür wird eine besonders intensive Behandlung durch Ärzte und Pflegekräfte benötigt.

Technische Fortschritte und die Digitalisierung ermöglichen es, mehr Informationen über jeden Patienten zu erfassen und zu verarbeiten als je zuvor. Ärzte, Pflegekräfte, aber auch Angehörige werden so mit großen Mengen an Informationen konfrontiert. Die erfolgreiche Kommunikation zwischen Behandelnden stellt eine wichtige Vorraussetzung für das Patientenoutcome dar. \cite{marxIntensivmedizin2015c} bezeichnen Kommunikationsprobleme als wichtigen Faktor für erhöhte Krankenhausmortalitätsraten. Die Autoren beschreiben weiter, dass bis zu 50 \% der klinisch relevanten Informationen, die noch in der Morgenvisite zwischen Ärzten ausgetauscht werden, schon in der Spätvisite des gleichen Tages nicht mehr übermittelt werden. %(Kap. 11.5.4)

Die Frage der effektiven Datenerfassung auf Intensivstationen ist somit eine über Leben und Tod. Dennoch kommt es aus unterschiedlichen Gründen vor, dass Informationen über den Gesundheitszustand der Patienten ungenau oder in zu geringem Umfang digital erfasst werden. 

Gegenstand der vorliegenden Arbeit ist ein Versuch, mittels maschinellem Lernen einen Beitrag zur Lösung dieses Problems beizutragen.

\subsection{medizinische Scores} \label{section:scores}

In \textit{Die Intensivmedizin} \citep{marxIntensivmedizin2015c} ist der Begriff des Scores folgendermaßen definiert:

\begin{itquote}
    "Ein Score ist der Versuch, eine komplexe klinische Situation auf einen eindimensionalen Punktwert abzubilden. Eine solche Reduktion verfolgt das Ziel, übergreifende Aspekte wie Schweregrad oder Prognose als Kombination einzelner Fakten objektiv zu fassen, um sie dann in unterschiedlichen Kollektiven vergleichend darstellen zu können."
\end{itquote}

Es handelt sich bei einem Score häufig um die Kombination mehrerer erfassbarer Werte, beispielsweise der Herzfrequenz oder dem Sauerstoffgehalt im Blut. Auch allgemeine Informationen über den Patienten wie das Alter oder bekannte Vorerkrankungen können berücksichtigt werden. Die Bestimmung eines Scores stellt also den Versuch dar, die komplexe, individuelle Situation eines Patienten auf einen numerischen Wert zu reduzieren. Dabei gehen unweigerlich Informationen verloren. Gleichzeitig erlaubt es die Erfassung von derartigen standardisierten Scores aber, auf einen Blick wichtige Informationen über den Zustand des Patienten zu erfassen. Durch eine derartige Reduktion auf das Wesentliche wird ferner ermöglicht, den pathologischen Verlauf eines Patienten über einen längeren Zeitraum zu analysieren, oder die Symptomatik mehrerer Patienten leichter miteinander zu vergleichen. Ein weiterer Vorteil ist es, dass, unter Voraussetzung der richtigen Anwendung, die Vergabe von Scores weitestgehend unabhängig von der subjektiven Einschätzung des Arztes oder der Pflegekraft erfolgt \citep{marxIntensivmedizin2015c}.
Eine Ausnahme unter den in den Datensätzen erfassten Werte bildet der CAM-ICU (siehe Abschnitt \ref{section:vorliegende_daten}). Das Ergebnis fällt hierbei entweder positiv oder negativ aus und stellt damit keinen Score im eigentlichen Sinne dar.
%Die Frage, ob es sich bei der Vorhersage der im Rahmen dieser Arbeit behandelten Scores um ein Regressions- oder ein Klassifikationsproblem handelt, wird in Abschnitt \ref{section:regrvsclf} weiter vertieft. 

\subsection{Ziel der Arbeit}
Das Ziel der vorliegenden Arbeit ist es, mit Hilfe von maschinellem Lernen ein statistisches Modell zu entwickeln, um anhand von Freitexten medizinische Scores möglichst akkurat vorherzusagen. Die Entwicklung eines solchen Modells ermöglicht es unter anderem, die tatsächlich eingetragenen Werte mit den Vorhersagen des Modells zu vergleichen, um daraus Rückschlüsse über die Qualtät der Datenerfassung an der Charité zu treffen.

\section{Maschinelles Lernen}
Der Begriff Maschinelles Lernen kennzeichnet einen modernen Ansatz in der Forschung an künstlicher Intelligenz. Obgleich die Anfänge des maschinellen Lernens bereits mehrere Jahrzehnte zurückliegen, hat es sich erst im Zeitalter von Big Data und besonders leistungsstarken Rechnern zu einem der Forschungsschwerpunkte der Angewandten Informatik entwickelt. Algorithmen aus dem Bereich des Maschinellen Lernens ermöglichen es, anhand von Eingabedaten Vorhersagen über Eigenschaften noch unbekannter Daten zu treffen \citep{mitchellMachineLearning1997}. \cite{mitchellMachineLearning1997} definiert diesen Vorgang folgendermaßen:

\begin{itquote}
    {\foreignlanguage{english}{"A computer program is said to learn from experience E with respect to some class of tasks T and performance measure P, if its performance at tasks in T, as measured by P, improves with experience E."}}
\end{itquote}

"Lernen" bezeichnet hier also die Fähigkeit eines Computerprogramms, sich anhand von neuen Eingabedaten selbstständig anzupassen und zu verbessern. Dies erfolgt im Allgemeinen durch den Versuch, eine vorher festgelegte, wohldefinierte Verlustfunktion zu minimieren. Die zahlreichen dafür vorliegenden numerischen Optimierungsverfahren sind ein grundlegender Untersuchungsgegenstand in der Forschung am maschinellen Lernen. Bei der Vorhersage medizinischer Scores ist die mittlere absolute Abweichung (mean absolute error, MAE) ein Beispiel für eine solche Verlustfunktion. Diese gibt an, wie weit die Vorhersagen des Modells von den tatsächlichen Werten\footnote{Hierbei gilt die Annahme, dass die eingetragenen Werte auch tatsächlich der Realität entsprechen, siehe Abschnitt \ref{section:genauigkeit_der_daten}.} im Durchschnitt abweichen. Werden hier niedrige Werte erreicht ist dies ein geeignetes Indiz für die Fähigkeit des Modells, gute Vorhersagen zu treffen.

\subsection{Überwachtes Lernen}\label{section:supervised_learning}
Algorithmen aus dem Bereich des Überwachten Lernen (supervised learning) basieren auf Datensätzen, die sowohl Eingabedaten, als auch die entsprechenden Ausgabedaten enthalten \citep{russellArtificialIntelligenceModern2020}. Es liegen also Wertepaare vor, die es einem Modell ermöglichen, Vorhersagen über die dazugehörigen Ausgaben anhand eine

Neben dem überwachten Lernen existieren weitere Arten des maschinellen Lernens, die aber nicht weiter Gegenstand dieser Arbeit sind.
%Ein- und Ausgabedaten liegen als Menge von 2-Tupeln vor: ${(a_n,b_n)} n,m \in \mathbb{N}, \forall n \in \mathbb{N}: a_n \in \mathbb{R}^m, b_n \in \mathbb{R} $ 

Die Verarbeitung natürlicher Sprache stellt eine besondere Herausforderung für das maschinelle Lernen dar.
Bevor ein Modell trainiert werden kann muss eine angemessene numerische Repräsentation der Eingabetexte gefunden werden. 

\subsection{Regression vs Klassifikation}\label{section:regrvsclf}
Probleme aus dem Bereich des überwachten maschinellen Lernens lassen sich im Allgemeinen in eine von zwei Kategorien einordnen:
Klassifizierung bezeichnet den Prozess, bei dem ein Datensatz einer oder mehreren Klassen aus einer endlichen Liste möglicher Klassen zugeordnet wird. Dieser Ansatz findet beispielsweise bei der automatischen Kategorisierung von E-Mails (Spam oder nicht Spam) oder bei der Erkennung von handschriftlichen Texten (welches Symbol aus einem gegebenen Alphabet ist dargestellt?) Anwendung (CITE). 
Da es sich bei den betrachteten medizinischen Scores um diskrete, ganzzahlige Werte aus einem endlichen Wertebereich handelt, liegt auch hier die Anwendung eines Klassifizierungs-Verfahrens nahe.

Betrachtet man aber die verschiedenen möglichen Werte eines Scores als separate und voneinander unabhängige Klassen, so ginge eine wichtige Information über deren Anordnung verloren. Bei den im Rahmen dieser Arbeit behandelten Scores handelt es sich stets um eindimensionale, metrische Skalen. Im mathematischen Sinne stellen sie Totalordnungen dar: Sie erfüllen also die Anforderungen der Reflexivität, Antisymmetrie, Transitivität und Totalität. Bezeichne $M$ die Menge aller möglichen Werte eines beliebigen medizinischen Scores. Es gilt also für alle $a,b,c \in M$:

\begin{equation*}
    \centering
    \begin{aligned}[c]
        a \leq a\\
        a \leq b \land b \leq a \; \Rightarrow \; a=b\\
        a \leq b \land b \leq c \; \Rightarrow \; a \leq c\\
        a \leq b \lor b \leq a
    \end{aligned}
    \qquad
    \begin{aligned}[c]
        \text{(Reflexivität)}\\
        \text{(Antisymmetrie)}\\
        \text{(Transitivität)}\\
        \text{(Totalität)}
    \end{aligned}
\end{equation*}

Damit lassen sich die verschiedenen Scores vergleichen und in ein Verhältnis setzen. So ist ein RASS-Wert\footnote{Richmond Agitation-Sedation Scale} von $-4$ (tief sediert) beispielsweise deutlich näher an $-3$ (mäßig sediert) als an $+1$ (unruhig). Bei gängigen Verfahren zur Klassifizierung ginge diese Information verloren, da bei Kenngrößen zur Bewertung solcher Modelle nur betrachtet werden kann, ob ein gegebener Eingabetext genau der richtigen Kategorie (dem richtigen Score) zugeordnet wurde oder nicht. 

Bei der vorliegenden Arbeit habe ich mich demnach dafür entschieden, die Vergabe von Scores anhand von Eingabetexten als klassisches Regressionsproblem zu betrachten, und die Ausgaben der Modelle im Zweifelsfall auf den nächstmöglichen ganzzahligen Wert zu runden. Dieser Ansatz fand auch bei früheren Arbeiten, die sich mit ähnlichen Fragestellungen befassten, Anwendung (CITE). Die Abweichung des vorhergesagten Werts eines Modells von dem nächst möglichen Wert stellt somit sogar einen rudimentären Ansatz zur Bewertung der Konfidenz bei der Vergabe einzelner Werte dar. %siehe: https://stats.stackexchange.com/a/282890

\subsection{Maschinelles Lernen in der Medizin}
Hier andere Einsatzgebiete von ML in Medizin \citep{krishnanSupervisedLearningApproach2018}. Und: Warum ML für unsere Problematik?

Problem, insbesondere in der Medizin: Explainable AI. Insbesondere bei ANN: Reihe von entscheidungen die von künstlichen neuronalen Netzwerken getroffen werden und zu einer bestimmten ausgabe führen sind für die Entwicker oft nicht nachvollziehbar (CITE). Das Teilgebiet XAI beschäftigt sich mit methoden, ML besser nachvollziehbar zu machen.