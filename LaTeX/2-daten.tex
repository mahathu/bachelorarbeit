% \section{Datenerfassung an der Charité}
Auf den Intensivstationen der Charité wird das Patientendatenmanagementsystem COPRA eingesetzt, um Informationen wie Patienten- und Behandlungsdaten zu erfassen, darzustellen und zu verarbeiten. Die Datensätze der vorliegenden Arbeit wurden beim Export aus dem System pseudonymisiert, indem patientenbezogene Daten wie Name und Geburtsdatum entfernt wurden. Insbesondere werden Zeitpunkte nicht in absoluter Form angegebenen, sondern relativ zum Beginn des Krankenhausaufenthalts des Patienten. Rückschlüsse auf die Identität der Patienten, deren Daten im Rahmen dieser Arbeit betrachtet werden, sind somit ausgeschlossen. Eine Zuordnung der erfassten Werte und Visitentexte untereinander zum selben Patienten bleibt aber durch eine den Patienten eindeutig identifizierenden Zahlenkombination weiterhin möglich.

Nach dem Export lagen die Daten in Form mehrerer Datendateien vor. \texttt{patienten.csv} und \texttt{delir.csv} enthalten Metainformationen über die betrachteten Patienten. Die beiden Dateien \texttt{scores1.csv} und \texttt{scores2.xlsx} geben Aufschluss über die erfassten Scores und eingetragenen Freitexte. Jede Zeile enthält hierbei jeweils die VarID, den betreffenden Patienten, den Zeitpunkt der Eintragung sowie den eigentlichen erfassten Wert. Bei der VarID handelt es sich um einen ganzzahligen Wert, mit der jede Art von auf der Intensivstation erfasstem Wert bzw. eingetragenem Visitentext Charité-intern repräsentiert und eindeutig identifiziert wird \reftbl{table:varids}.

\section{vorliegende Datensätze} \label{section:vorliegende_daten}
Der mir für diese Arbeit zur Verfügung gestellte Datensatz umfasst insgesamt 1357 Patienten-Aufenthalte auf den Intensivstationen der Charité Berlin aus dem Jahr 2017. Jeder Aufenthalt wird über eine numerische, inkrementell aufsteigende Nummer\footnote{in den exportierten Datensätzen als n\_ID bezeichnet} eindeutig identifiziert. Ein Patient, der im Laufe seiner Behandlung mehrere Male auf eine Intensivstation verlegt wird, erhält für jeden separaten Aufenthalt eine neue Identifikationsnummer. Für jeden Patient liegen Informationen über Geschlecht, BMI\footnote{Body-Mass-Index}, Alter zum Zeitpunkt der Aufnahme und ob während des Aufenthalts ein Delir\footnote{F05.* gemäß ICD-10} diagnostiziert wurde vor. Weiterhin ist die Gesamtdauer des Aufenthalts auf der Intensivstation vermerkt, sowie ob der Patient während seines Aufenthalts verstorben ist. Bei den vorliegenden Patienten lag die Mortalität während des Aufenthalts bei etwa 17\% ($n=225$). Die beobachteten Aufenthalte verliefen über Zeiträume von wenigen Stunden bis zu mehreren Monaten. Die mittlere Aufenthaltsdauer während des Beobachtungszeitraums liegt bei etwa 7,2 Tagen, der Median beträgt 3,2 Tage. Fast die Hälfte aller Aufenthalte endete also nach weniger als drei Tagen. Nur etwas mehr als ein Viertel der Patienten ($n=375$) wurden für eine Woche oder länger auf der Intensivstation behandelt.

Für jeden Patienten werden während seines Aufenthalts medizinische Scores \refsec{section:scores} bestimmt sowie Visitentexte geschrieben. Die Visitentexte werden digital erfasst und liegen im Unicode-Zeichensatz vor, folgen allerdings im Allgemeinen keinem einheitlichen Muster. Es handelt sich also um unstrukturierte Freitexte, und es liegt im Ermessen der behandelnden Ärzte bzw. Pflegekräfte, einen aussagekräftigen Text zu formulieren. Ebenso können sich Faktoren wie Zeitdruck, Stress oder Ablenkungen durch die Umgebung negativ auf Umfang und Genauigkeit der eingetragenen Texte auswirken \citep{marxIntensivmedizin2015c}.
%Bei den erfassten Scores handelt es sich um medizinische Bewertungssysteme, bei denen die Verfassung des betrachteten Patienten anhand klar definierter Regeln anhand einer Punktzahl bewertet wird(CITE). Häufig beziehen sich die Scores dabei nur auf einen kleinen Teil der Gesamtverfassung des Patienten, beispielsweise auf den Grad der Sedierung (RASS) oder auf das Schmerzempfinden, sofern dieses nicht vom Patienten selber mitgeteilt werden kann (BPS).
%Die in der vorliegenden Arbeit betrachteten Scores weisen es sich um Scores mit einem hohen Grad an Spezifität. Das Ziel solcher Scores ist es nicht, den Gesundheitszustand eines Patienten in seiner Gesamtheit zu erfassen, sondern eher einen klar abgrenzbaren Teilbereich davon \citep{marxIntensivmedizin2015c}. 

\begin{table}[h] %here > top > bottom
    \centering
    \begin{tabular}{llr}\toprule
        \textbf{VarID}	&\textbf{Name} &\textbf{Wertebereich} \\\midrule
        20512769    & Glasgow Coma Scale (GCS)          & $v \in \interval{3}{15}$ \\
        20512801    & Behavior Pain Scale (BPS)         & $v \in \interval{3}{12}$ \\
        20512802    & Delirium Detection Score (DDS)    & $v \in \interval{3}{35}$ \\
        22085815    & Visite\_ZNS                        & Freitext \\
        22085820    & Visite\_Oberarzt                   & Freitext \\
        22085836    & Visite\_Pflege                     & Freitext \\
        22085897    & Ramsay Sedation Scale             & $v \in \interval{0}{6}$ \\
        22085911    & NRS/VAS (Visual Analogue Scale)   & $v \in \interval{0}{10}$ \\
        22086067    & Vigilanz                          & Freitext \\
        22086158    & Richmond Agitation Sedation Scale (RASS) & $v \in \interval{-5}{4}$ \\
        22086169    & CAM-ICU                           & $v \in \{	\text{neg.}, \text{pos.}, \text{unmögl.} \}$ \\
        22086170    & BPS-Bewertung                     & Freitext \\
        22086172    & NRS/VAS Bedingungen               & Freitext \\\bottomrule
    \end{tabular}

    \caption{Übersicht aller erfassten Scores und Freitexte}
    \label{table:varids}
\end{table}

Tabelle \ref{table:varids} enthält eine Übersicht über alle Scores und Freitexte, die in dem gegebenen Datensatz erfasst wurden. Es folgt eine detailliertere Beschreibung derjenigen Werte, die für den Inhalt dieser Arbeit besondere Relevanz haben:

\paragraph{Glasgow Coma Scale}
Bei der Glasgow Coma Scale (GCS) wird die Schwere einer möglicherweise vorliegenden Bewusstseinsstörung anhand von drei Kategorien (Augenöffnen, verbale Antwort und motorische Reaktion) ermittelt. In jeder Kategorie wird eine Punktzahl ermittelt, die dann zu einem Endergebnis aufsummiert werden. Insgesamt sind so Werte zwischen einschließlich 3 und 15 möglich \citep{teasdaleAssessmentComaImpaired1974,marxIntensivmedizin2015c}.

\paragraph{RASS}
Die Richmond Agitation Sedation Scale (RASS) ist eine zehnstufige Messzahl, die den Grad der Sedierung eines Intensivpatienten beschreibt. Mögliche Werte liegen zwischen $-5$ ("unarousable"/"nicht erweckbar") und $4$ ("combative"/"streitlustig"). Bei den betrachteten Patienten wurde der Wert $0$ ("aufmerksam und ruhig") am häufigsten erfasst, was dem erwünschten Wert entspricht, solange keine Sedierung indiziert ist. Die Autoren der Skala empfehlen eine Reihe von Stimuli, die dem Patienten präsentiert werden sollen, um eine besonders genaue Bestimmung des richtigen Wertes zu ermöglichen \citep{sesslerRichmondAgitationSedation2002a}. Die RASS weist eine hohe Reliabilität und Validität auf und gilt im deutschsprachigen Raum als Goldstandard zum Monitoring der Sedierungstiefe \citep{marxIntensivmedizin2015c, muellerAnalgesieSedierungUnd2015}.

% \paragraph{CAM-ICU}
% bla bla bla %https://www.ai-online.info/images/ai-ausgabe/2009/10-2009/2009_10_592-600_Confusion%20Assessment%20Method%20for%20Intensive%20Care%20Unit%20zur%20routine-maessigen%20Kontrolle%20.pdf
% Eine Besonderheit des CAM-ICU unter den vorliegenden Daten ist, dass es sich hierbei um einen binären Wert handelt:)

\paragraph{Visite\_ZNS, Visite\_Oberarzt und Visite\_Pflege}
Visitentexte unterschiedlicher Länge \reffig{fig:text_hist}, die in regelmäßigen Abständen von Ärzten oder Pflegepersonal verfasst werden und Informationen zum Gesundheitszustand des betrachteten Patienten beinhalten.

\begin{figure}[htb]
    \captionsetup{justification=centering}
    \centering
    \includegraphics[width=.67\textwidth]{words_density}
    \caption{Relative Länge der verschiedenen Visitentext-Kategorien. Zur besseren Vergleichbarkeit wurde die x-Achse auf 250 Wörter begrenzt.}
    \label{fig:text_hist}
\end{figure}

\begin{figure}[htb]
    \captionsetup[subfigure]{labelformat=empty, justification=centering}

    \centering
    \subfloat[Behavior Pain Scale]{\includegraphics[width=0.35\textwidth]{score_histograms/hist_Behavior_Pain_Scale}} \qquad
    \subfloat[CAM-ICU]{\includegraphics[width=0.35\textwidth]{score_histograms/hist_CAM-ICU}} \\ % linebreak here
    % \subfloat[Delirium Detection Score]{\includegraphics[width=4.75cm]{score_histograms/hist_Delirium_Detection_Score}} \\ % linebreak here
    \subfloat[Glasgow Coma Scale]{\includegraphics[width=0.35\textwidth]{score_histograms/hist_Glasgow_Coma_Scale}} \qquad
    \subfloat[Richmond Agitation Sedation Scale]{\includegraphics[width=0.35\textwidth]{score_histograms/hist_Richmond_Agitation_Sedation_Scale}}
    % \subfloat[Visual Analogue Scale]{\includegraphics[width=4.75cm]{score_histograms/hist_Visual_Analogue_Scale}} \\
    \caption{Histogramme einiger erfassten Scores}
    \label{fig:score_histograms}
\end{figure}

% \subsection{Exemplarische Vorstellung eines Patienten} %z.b. 0430
% Aufenthalt des Patienten hier detailliert beschreiben, und seinen Scatterplot einfügen (aber noch ohne Pfeile).

\section{Generierung von Schlüssel-Wert-Paaren}\label{sec:pairgen}
Eine besondere Herausforderung stellte dar, dass die verschiedenen Scores und Texte zu unterschiedlichen Zeiten und unabhängig voneinander eingetragen werden (siehe Abbildung \ref{fig:pat40_scatter}).
%Die Informationen über Zeitpunkt, Art der Eintragung (VarID) sowie der eigentliche Wert liegen jeweils in Form von Tripeln in den entsprechenden Log-Dateien vor. Bei den Visitentexten enthalten die Tripel den entsprechenden Text als Wert der Eintragung.
Um ein Machine-Learning Modell mit einer hohen Dimensionalität der Eingabevektoren zu trainieren ist eine hohe Anzahl von Trainingspaaren notwendig. Im konkreten Fall dieser Arbeit enthält jedes Wertepaar einen Visitentext sowie einen medizinischen Score, der den in dem Text angegebenen Informationen über die Verfasssung des Patienten möglichst genau entsprechen soll. Zusammen bilden diese Paare die Grundlage der Modelle, selbstständig noch nicht gesehene Texte bewerten zu können. Aufgrund der zeitlichen Unterschiede zwischen den Eintragungen von Texten und Scores erwies sich allerdings ebenjene Zuordnung zueinander als nicht trivial. 

\begin{figure}[htb]
    \captionsetup{justification=centering}
    \centering
    \includegraphics[width=\textwidth]{patienten_scatters/patient_0040_noarrows.png}
    \caption{Die Visitentexte (blau) und Scores (orange) werden regelmäßig, aber zeitlich unabhängig voneinander eingetragen bzw. erfasst.}
    \label{fig:pat40_scatter}
\end{figure}

Ein möglicher naiver Ansatz wäre es, für einen bestimmten Zeitraum den Mittelwert aller Erhebungen eines Scores zu ermitteln, sowie alle Visitentexte, die in diesem Zeitraum erfasst wurden, aneinander zu reihen. Wählt man beispielsweise einen Zeitraum von 4 Stunden, erhielte man so für jeden Patient, Visitentext und Score 6 Wertepaare pro Tag, mit denen sich ein statistisches Modell trainieren ließe.

Aus zwei Gründen habe ich mich allerdings für einen anderen Ansatz entschieden: Zum einen ist es durchaus möglich, dass sich ein bestimmter Wert innerhalb von wenigen Sekunden stark verändert -- beispielsweise die RASS, wenn die Sedierung eines Patienten eingeleitet wird. Derartige Informationen könnten verloren gehen, wenn man von vornherein nur feste Zeiträume betrachtet. Weiterhin gibt es zwischen individuellen Patienten hohe Schwankungen in der Frequenz und Häufigkeit, in denen bestimmte Scores erfasst werden (siehe Abbildung \ref{fig:pat_example_scatterplots}). Würde man einen festen Zeitraum bestimmen, in welchem die Werte stets zusammengefasst werden, so würden unter Umständen viele Informationen verloren gehen. Auf der anderen Seite kann es vorkommen, dass ein Score in einem Zeitraum gar nicht oder sehr selten erfasst wurde. Die Gefahr, dass sich die Erhebung eines Score dann auf einen ganz anderen Zeitpunkt als ein im gleichen Zeitraum geschriebener Text bezieht, wäre damit sehr groß. Die Korrelation zwischen Inhalt des Visitentexts und des Scores wäre womöglich sehr gering, und die entstehenden Wertepaare dementsprechend ungeeignet, ein effektives Modell zu trainieren.

Zur Generierung der Wertepaare habe ich mich demnach für eine andere Methodik entschieden. Die Implementierung liegt hierfür in \texttt{skripte/find\_training\_pairs.py} vor. Jeder Erhebung eines Scores wird derjenige Visitentext zugeordnet, der zeitlich betrachtet am nächsten zu dem erhobenen Score eingetragen wurde. Dabei kann der Text sowohl vor als auch nach der Eintragung des Scores verfasst worden sein. Weiterhin werden \texttt{Visite\_Pflege}, \texttt{Visite\_ZNS} und \texttt{Visite\_Oberarzt} unabhängig voneinander betrachtet, sodass pro Eintragung eines Scores jeweils eine Zuordnung zu jedem der drei Texte erfolgt. In Abbildung \ref{fig:pat_example_scatterplots} sind diese Zuordnungen durch Pfeile dargestellt. Beim späteren Training der Modelle werden dann nur solche Trainingspaare betrachtet, die den gewünschten Eingabetext beinhalten. Außerdem kann es auch mit dieser Methode vorkommen, dass in einigen Ausnahmefällen die Eintragungen von Score bzw. Text mit einem großen zeitlichen Abstand erfolgt sind. Deswegen wird für jedes Wertepaar zusätzlich die zeitliche Differenz der beiden Eintragungen gespeichert. Das ermöglicht, dass vor dem Trainining eines Modells nur solche Wertepaare selektiert werden, deren zeitlicher Abstand unter einem festgelegten Maximalwert liegt. Da zunächst alle erdenklichen Wertepaare vorliegen kann dieser maximale Abstand je nach Bedarf besonders groß oder klein gewählt werden, beispielsweise dann, wenn ein Modell eine besonders hohe Anzahl an Trainingspaaren benötigt. Abbildung \ref{fig:n_pairs_by_max_offset} zeigt die Anzahl der Wertepaare pro Kategorie von Eingabetext je nach maximal erlaubter Zeit zwischen den beiden Eintragungen.

\begin{figure}[htp]
    \captionsetup{justification=centering}
    \centering
    \subfloat{{\includegraphics[width=7cm]{pairs_by_max_min_RASS_FIXED.png} }}
    \quad
    \subfloat{{\includegraphics[width=7cm]{pairs_by_max_min_GCS_FIXED.png} }}
    \caption{Anzahl Wertepaare bei maximal zulässiger Differenz zwischen \mbox{Zeitpunkten}}
    \label{fig:n_pairs_by_max_offset}
\end{figure}
% Möglichkeit das noch zu verbessern: statt alle texte innerhalb von z.b. 6h eines Textes zu betrachten: solche ausschließen, die näher an einer anderen erfassung des gleichen scores liegen? 

% Hier Bild mit 4 Patienten-Scatterplots! Bildbeschreibung: "Patient X ist der Patient, dessen Aufenthalt in Sektion 2.2.2 detailliert vorgestellt wurde."
\begin{figure}[p]
    \centering
    \subfloat{\includegraphics[height=1.87in]{patienten_scatters/patient_0040}} \\
    \subfloat{\includegraphics[height=1.87in]{patienten_scatters/patient_0182}} \\
    \subfloat{\includegraphics[height=1.87in]{patienten_scatters/patient_0316}} \\
    \subfloat{\includegraphics[height=1.87in]{patienten_scatters/patient_0364}} 
    \caption{Übersicht der erfassten Werte einiger Patienten}
    \label{fig:pat_example_scatterplots}
\end{figure}

\section{Genauigkeit der erfassten Daten}\label{section:genauigkeit_der_daten}
Neben einer möglichst genauen, automatischen Vergabe von medizinischen Scores anhand von Visitentexten stellt sich weiterhin die Frage, welche der gegebenen Eingabewerte überhaupt der Wahrheit entsprechen, beziehungsweise diese am genauesten abbilden. Die Modelle, die in Abschnitt \ref{chap-vorgehensweise} beschrieben werden, nehmen die eingetragenen Scores als "absolute Wahrheit"\footnote{Im maschinellen Lernen und in verwandten Bereichen auch als "ground truth" bezeichnet} an. Wir gehen also davon aus, dass diese Eintragungen die Kondition eines Patienten im Zweifelsfall genauer beschreiben als die Visitentexte, die zu einem ähnlichen Zeitpunkt verfasst wurden. Dies stellt allerdings eine Vereinfachung der Situation dar. Im Allgemeinen ist es fast unmöglich, nachträglich zu bestimmen, welcher Wert die Realität widerspiegelt, wenn Score und Text voneinander abweichen. 

Diese Art von Widerspruch tritt in den betrachteten Datensätzen häufig auf und kann mehrere Gründe haben. Zum einen kann es vorkommen, dass Ärzte oder Pflegekräfte aufgrund von Zeitmangel oder Stress einen Text nicht mit der nötigen Gewissenhaftigkeit oder im nötigen Umfang verfassen, um aus den gegebenen Informationen einen Score herzuleiten. Im Extremfall kann es sogar vorkommen, dass Texte oder Scores aus der vorhergehenden Erhebung unverändert übernommen werden, auch wenn sich der Zustand des betroffenen Patienten in Wahrheit verändert hat. Wird aber gleichzeitig ein entsprechender Score bzw. Text aktualisiert kommt es zu einem Widerspruch. Ferner ist es selbst ohne menschlichen Fehler möglich, dass Score und Visitentext unterschiedliche Werte indizieren, auch wenn diese zeitlich nah beieinander liegen. Da sich insbesondere auf einer Intensivstation der Zustand eines Patienten sehr schnell ändern kann, kann es durchaus vorkommen, dass sich zwei zeitlich sehr nah beieinanderliegende Eintragungen inhaltlich stark unterscheiden, obwohl beide Werte zum Zeitpunkt ihrer Eintragung als realitätsgetreu angesehen werden können.

Diese Gründe sind nicht weiter Beobachtungsgegenstand der vorliegenden Arbeit. Dennoch muss diese Art von möglichen Unstimmigkeiten in den vorliegenden Daten bei der Konzeption, Entwicklung und Bewertung der Modelle beachtet werden. Insbesondere stellt die Qualität der Eingabedaten, ohne weitere Maßnahmen zur Datenbereinigung, womöglich eine obere Schranke für die Performance der Modelle dar.