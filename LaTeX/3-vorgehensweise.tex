Für die Konzeption eines geeigneten Modells zur Vorhersage medizinischer Scores habe ich mich dafür entschieden, zwei separate, voneinander unabhängige Ansätze zu verfolgen. In Abschnitt \ref{section:baseline} wird die Entwicklung eines einfachen Regressionsmodells anhand einer sogenannten Support Vector Machine beschrieben. Abschnitt \ref{section:ELM} beinhaltet die Entwicklung eines Modells basierend auf der Extreme Learning Machine, einem Verfahren, das es ermöglicht, künstliche neuronale Netze mit nur einem hidden layer effizient zu trainieren \citep{huangExtremeLearningMachine2006}. Weiterhin werden dort weitere Methoden zur Vorverarbeitung der Eingabedaten vorgestellt sowie deren Wirkung bewertet. 

\section{Baseline-Modell}\label{section:baseline}

\subsection{Datenaufbereitung}

Die Verarbeitung natürlicher Sprache stellt eine besondere Herausforderung für das maschinelle Lernen dar. Herkömmliche Modelle aus dem Bereich des überwachten Lernens können nur anhand von numerischen Eingabedaten endlicher Länge trainiert werden.
Das bedeutet, dass für jeden Eingabetext eine angemessene numerische Repräsentation gefunden werden muss, bevor das Modell trainiert werden kann. Dieser Prozess, bei dem jeweils ein Eingabewert auf einen numerischen Vektor gleicher Länge transformiert wird (den sog. \textit{feature vector}), wird im Allgemeinen als \textit{featurization} bezeichnet. Das Ziel ist hierbei, die für den Ausgabewert relevanten Informationen möglichst akkurat abzubilden. Informationen, die bei der Vorhersage der Ausgabewerte irrelevant sind, sollten nicht teil des feature vectors sein, um ein overfitting durch zu viele Daten zu verhindern (siehe Abschnitt \ref{section:overfitting}).

Ein trivialer Ansatz für die featurization von Eingabetexten endlicher Länge ist das sogenannte Bag-of-Words-Modell. Hierbei wird zunächst das Vokabular, d.h. eine Liste jeglicher in allen Texten auftretenden Wörter, ermittelt. Auch Wörter, die mehrmals vorkommen, 

Hierbei wird zunächst jeder Eingabetext in die einzelnen Wörter unterteilt. Jedes Wort, das in dem gesamten Datensatz mindestens ein mal vorkommt, wird mit einer individuellen Nummer versehen. Die Länge der feature-Vektoren der einzelnen Eingabetexte entspricht der Gesamtanzahl

einfaches beispiel: bag of words. (spezialfall: bag of ngrams, indem der begriff "wort"/"token" verallgemeinert wird.)
erweiterung: tf. erweiterung davon: tfidf
1) vectorization
2) tfidf transformiert jeden eingabetext in einen vector mit gleicher länge. länge=anzahl der unique words.
\subsection{Modell}

Was ist ein SVR?
scikit learn \citep{JMLR:v12:pedregosa11a}.
finden der besten hyperparameter mit hilfe von GridSearchCV.

\subsection{Ergebnisse}
Bla, hier ergebnisse. erkenntnis: besonders wichtig war verarbeitung der eingabedaten, nicht modell an sich.

\section{künstliches neuronales Netzwerk}\label{section:ELM}
\subsection{Vorverarbeitung der Eingabedaten}
hier fokus darauf, wie ich eingabedaten verbessert habe. sonderzeichen entfernt, stopwörter entfernt. dabei darauf geachtet: keine vorgefertigte liste von stopwörtern benutzt, weil diese auf gestemmten text ausgelegt sind und auch wörter wie "kein" enthalten, die für meinen text relevant sind.


Rechtschreibprüfung: wie viele wörter in den jeweiligen texten, wie viele unique? welches sind die häufigsten? Bla

% bei spell checker eher für einen naiven ansatz entschieden:
% wörter, die öfter als (3) mal vorkommen werden schon als
% richtig angesehen, auch wenn da viele falsch geschriebene
% wörter mit dabei sind. würde man aber nur wörter ansehen
% die noch öfters vorkommen, würde man auch wörter korriegieren
% die eigentlich richtig sind, nur weil sie nicht oft genug
% vorkamen, um als richtig angesehen zu werden:
% 'spannung' > spaltung
% 'verformung' > versorgung
Nur offensichtliche tippfehler korrigiert.
Beschreiben: Verbesserung der featurization der eingabedaten durch word2vec. verbesserung auch durch spell correction, weil es viele falsch geschriebene wörter gibt, die dann in word2vec gar nicht auftauchen. (zeigen, wie sich die anzahl der unbekannten wörter durch spell correction verringert!)

Vergleich unbekannte/bekannte wörter mit spell correction, substitutions etc und ohne.

\subsection{Vektorisierung mittels word2vec}
hier: wie funktioniert w2v, was ist performance von svm aus abschnitt 3.1 mit word2vec?

\subsection{Extreme Learning Machine}
was ist eine ELM \citep{huangExtremeLearningMachine2006}, was für ne toolbox hab ich verwendet \citep{akusokHighPerformanceExtremeLearning2015}

\subsection{Ergebnisse}
Das modell wurde anhand des in Abschnitt 3.2.1 beschriebenen kreuzvalidierungsverfahren getestet. bla.