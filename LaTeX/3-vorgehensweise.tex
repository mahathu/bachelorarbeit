Bei der Konzeption eines geeigneten Modells zur Vorhersage medizinischer Scores habe ich mich dafür entschieden, zwei separate, voneinander unabhängige Ansätze zu verfolgen. In Abschnitt \ref{section:baseline} wird die Entwicklung eines einfachen Regressionsmodells auf Basis der sog. Support Vector Machine beschrieben. Abschnitt \ref{section:ELM} beinhaltet die Entwicklung eines Modells basierend auf der Extreme Learning Machine, einem Verfahren, das es ermöglicht, künstliche neuronale Netze mit nur einem hidden layer effizient zu trainieren \citep{huangExtremeLearningMachine2006}. Weiterhin werden in Abschnitt \ref{section:ELM} weitere Methoden zur Vorverarbeitung der Eingabedaten vorgestellt sowie deren Wirkung bewertet. 

\section{Baseline-Modell}\label{section:baseline}
\subsection{Datenaufbereitung}
Die Verarbeitung natürlicher Sprache stellt eine besondere Herausforderung für das maschinelle Lernen dar.
Bevor ein Modell trainiert werden kann muss eine angemessene numerische Repräsentation der Eingabetexte gefunden werden. 

1) vectorization

2) tfidf transformiert jeden eingabetext in einen vector mit gleicher länge. länge=anzahl der unique words.
\subsection{Modell}
scikit learn \citep{JMLR:v12:pedregosa11a}
\subsection{Ergebnisse}
\section{Extreme Learning Machine}\label{section:ELM}

\subsection{Rechtschreibprüfung}
wie viele wörter in den jeweiligen texten, wie viele unique? welches sind die häufigsten? Bla

% bei spell checker eher für einen naiven ansatz entschieden:
% wörter, die öfter als (3) mal vorkommen werden schon als
% richtig angesehen, auch wenn da viele falsch geschriebene
% wörter mit dabei sind. würde man aber nur wörter ansehen
% die noch öfters vorkommen, würde man auch wörter korriegieren
% die eigentlich richtig sind, nur weil sie nicht oft genug
% vorkamen, um als richtig angesehen zu werden:
% 'spannung' > spaltung
% 'verformung' > versorgung
Beschreiben: Verbesserung der featurization der eingabedaten durch word2vec. verbesserung auch durch spell correction, weil es viele falsch geschriebene wörter gibt, die dann in word2vec gar nicht auftauchen. (zeigen, wie sich die anzahl der unbekannten wörter durch spell correction verringert!)

Vergleich unbekannte/bekannte wörter mit spell correction, substitutions etc und ohne.
% * numerische optimierung: Lösungsraum, Bewertungsfunktion
%   * das Gradientenverfahren
%   * stochastic gradient descend
% * Evaluation von ML-Modellen
%       * z.b. cross validation
%       * vermeiden: data leakage

%bla bla bla. "das modell wurde anhand des in Abschnitt 3.2.1 beschriebenen kreuzvalidierungsverfahren getestet."
