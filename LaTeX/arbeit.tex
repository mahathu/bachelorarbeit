\documentclass[
	a4paper, % Papierformat A4
	12pt, % Schriftgröße 12pt
	twoside, % + BCOR darunter: für doppelseitigen Druck aktivieren, sonst beide deaktivieren
	BCOR=5mm, % Dicke der Bindung berücksichtigen (Copyshop fragen, wie viel das ist)
	fleqn, % flush equations left (--> Formeln linksbündig)
	bibliography=totoc, %bibliographie im Inhaltsverzeichnis
	headinclude, %scheint keinen Unterschied zu machen
	headsepline
]{scrreprt}

\usepackage{pdfpages}

\usepackage{float}
\usepackage[caption = false]{subfig}
\usepackage{graphicx}
\graphicspath{ {fig/} }

\usepackage{xcolor}
\usepackage[english, main=ngerman]{babel} % englisch für Zitate auf Englisch
\usepackage[onehalfspacing]{setspace}
\usepackage[utf8]{inputenc}
\usepackage{microtype} % better kerning, word spacing etc

\usepackage[colorlinks=true,allcolors=blue]{hyperref} %change options to [hidelinks] for printing!!
\usepackage{booktabs} % schönere Tabellen

\usepackage{csquotes} % " wird zu Anführungszeichen
\MakeOuterQuote{"}

\newenvironment{itquote} % italics quote and " "
{\begin{quote}\itshape}
{\end{quote}}

\usepackage{amsmath} %z.b align
\usepackage{interval} %für Wertebereiche

\usepackage{footnote} %um fußnoten auch in Tabellen zu ermöglichen
\makesavenoteenv{tabular}
\makesavenoteenv{table}

\usepackage[square]{natbib} % scheinbar die beste Wahl für mich https://tex.stackexchange.com/a/25702/217229
\bibliographystyle{bibstyle.bst} % GUI to make bst style: latex makebst

% \usepackage{fontspec} % custom font (needs compiling via xelatex)
% \setmainfont{Fira Sans}

\RedeclareSectionCommand[beforeskip=0pt,afterindent=false]{chapter} %weniger Abstand nach oben bei Kapitelüberschriften

\begin{document}
	\renewcommand{\arraystretch}{1.15} %Zeilenabstand innerhalb von Tabellen

	% ========================================== %
	% Am Schluss noch überprüfen:                %
	% - einheitliche Zeitform                    %
	% - gendergerechte Sprache (bäh)             % 
	% ========================================== %


	% Deckblatt:
	\includepdf[pages={1}]{Deckblatt/deckblatt.pdf}

	% Oben auf der Seite aktuelles Kapitel hinschreiben
	\pagestyle{headings}

	% Abstract (muss für komascript so gemacht werden)
	\chapter*{Abstract}
	Hier kommt ein toller Abstract

	% Inhaltsverzeichnis:
	%\setlength{\parskip}{-em} % falls das Inhaltsverzeichnis eng wird
	\tableofcontents

	\setlength{\parskip}{.2em} % space between paragraphs. Erst nach dem Inhaltsverzeichnis:)

	\chapter{Einführung}
	\section{Datenerfassung auf Intensivstationen}
%   * worum geht es?
%   * Thematik
%     * Problematik der Datenerfassung auf ICUs

\cite{marxIntensivmedizin2015c} verorten die Intensivmedizin im Spannungsfeld zwischen Heilen und Sterben. Eine Verlegung auf die Intensivstation erfolgt häufig infolge einer besonders schweren oder lebensbedrohlichen Erkrankung oder Verletzung. Eine Behandlung auf der Intensivstation verfolgt, sofern möglich, das Ziel, den Patienten insofern zu kurieren, dass diesem ein Weiterleben unabhängig von den besonderen technischen und personellen Möglichkeiten der Intensivmedizin möglich ist. Dafür wird eine besonders intensive Behandlung durch Ärzte und Pflegekräfte benötigt.

Jüngere technische Entwicklungen ermöglichen es, mehr Informationen über jeden Patienten zu erfassen und zu verarbeiten als je zuvor. Ärzte, Pflegekräfte, aber auch Angehörige werden so mit großen Mengen an Informationen konfrontiert. Die erfolgreiche Kommunikation zwischen Behandelnden stellt eine wichtige Vorraussetzung für das Patientenoutcome dar. \cite{marxIntensivmedizin2015c} bezeichnen Kommunikationsprobleme als wichtigen Faktor für erhöhte Krankenhausmortalitätsraten. Die Autoren beschreiben weiter, dass bis zu 50 \% der klinisch relevanten Informationen, die noch in der Morgenvisite zwischen Ärzten ausgetauscht werden, schon in der Spätvisite des gleichen Tages nicht mehr übermittelt werden. %(Kap. 11.5.4)

Die Frage der effektiven Datenerfassung auf Intensivstationen ist also eine über Leben und Tod. Dennoch kommt es aus unterschiedlichen Gründen vor, dass Informationen über den Gesundheitszustand der Patienten ungenau oder in zu geringem Umfang digital erfasst werden. 

Gegenstand der vorliegenden Arbeit ist ein Versuch, mittels maschinellem Lernen einen Beitrag zur Lösung dieses Problems beizutragen.

\subsection{medizinische Scores} \label{section:scores}

In \textit{Die Intensivmedizin} \citep{marxIntensivmedizin2015c} ist der Begriff des Scores folgendermaßen definiert:

\begin{itquote}
    "Ein Score ist der Versuch, eine komplexe klinische Situation auf einen eindimensionalen Punktwert abzubilden. Eine solche Reduktion verfolgt das Ziel, übergreifende Aspekte wie Schweregrad oder Prognose als Kombination einzelner Fakten objektiv zu fassen, um sie dann in unterschiedlichen Kollektiven vergleichend darstellen zu können."
\end{itquote}

Es handelt sich bei einem Score häufig um die Kombination mehrerer erfassbarer Werte, beispielsweise der Herzfrequenz oder dem Sauerstoffgehalt im Blut. Auch allgemeine Informationen über den Patienten\footnote{Hier und im Rest der Arbeit umfasst das generische Maskulinum stets Personen beider Geschlechter.} wie das Alter oder bekannte Vorerkrankungen können berücksichtigt werden. Die Bestimmung eines Scores stellt also den Versuch dar, die komplexe, individuelle Situation eines Patienten auf einen numerischen Wert zu reduzieren. Dabei gehen unweigerlich Informationen verloren. Gleichzeitig erlaubt es die Erfassung von derartigen standardisierten Scores aber, auf einen Blick wichtige Informationen über den Zustand des Patienten zu erfassen. Durch eine derartige Reduktion auf das Wesentliche wird ferner ermöglicht, den pathologischen Verlauf eines Patienten über einen längeren Zeitraum zu analysieren, oder die Symptomatik mehrerer Patienten leichter miteinander zu vergleichen. Ein weiterer Vorteil ist es, dass, unter Voraussetzung der richtigen Anwendung, die Vergabe von Scores weitestgehend unabhängig von der subjektiven Einschätzung des Arztes oder der Pflegekraft erfolgt \citep{marxIntensivmedizin2015c}.
Eine Ausnahme unter den in den Datensätzen erfassten Werte bildet der CAM-ICU (siehe Abschnitt \ref{section:vorliegende_daten}). Das Ergebnis fällt hierbei entweder positiv oder negativ aus und stellt damit keinen Score im eigentlichen Sinne dar.
%Die Frage, ob es sich bei der Vorhersage der im Rahmen dieser Arbeit behandelten Scores um ein Regressions- oder ein Klassifikationsproblem handelt, wird in Abschnitt \ref{section:regrvsclf} weiter vertieft. 

\subsection{Ziel der Arbeit}
%Formulierung "Gegenstand der Arbeit" irgendwo verwenden
Das Ziel der vorliegenden Arbeit ist es, mit Hilfe von maschinellem Lernen ein statistisches Modell zu entwickeln, um anhand von Freitexten medizinische Scores möglichst akkurat vorherzusagen. Die Entwicklung eines solchen Modells ermöglicht es unter anderem, die tatsächlich eingetragenen Werte mit den Vorhersagen des Modells zu vergleichen, um daraus Rückschlüsse über die Qualtät der Datenerfassung an der Charité zu treffen.

\section{Maschinelles Lernen}
% Hier auf historischen Kontext eingehen! (big data, processing power Moore's law etc)

Der Begriff Maschinelles Lernen bezeichnet einen modernen Ansatz in der Arbeit an künstlicher Intelligenz. (cite).

hat sich im Zeitalter von Big Data und leistungsstarken Rechnern zu einem der Hauptforschungspunkte der Informatik entwicklt (cite). 

\citet{mitchellMachineLearning1997} definiert den Vorgang maschinellen Lernens folgendermaßen:

\begin{itquote}
    {\foreignlanguage{english}{"A computer program is said to learn from experience E with respect to some class of tasks T and performance measure P, if its performance at tasks in T, as measured by P, improves with experience E."}}
\end{itquote}

%In der vorliegenden Arbeit handelt es sich bei E um bereits gelabelte Wertepaare und bei T um die Aufgabe, einem noch unbekannten Text den richtigen Wert aus dem Wertebereich eines medizinischen Scores zuzuordnen. Für P kommen verschiedene Metriken in Frage, um die Qualität des Modells zu bemessen, beispielsweise das Bestimmtheitsmaß R^2 oder MAE (siehe Abschnitt 1.2.3 performance metrics :))

Im Allgemeinen werden also Algorithmen, die aus großen Datensätzen "lernen" und damit Vorhersagen über unbekannte Daten machen, als maschinelles Lernen bezeichnet.

\subsection{Überwachtes Lernen}\label{section:supervised_learning}
Beschreiben: Hier gehts um supervised ML! (neben supervised gibt es noch: unsupervised, wo foo, und Reinforcement learning, wo bar.)

Texte müssen in numerische Eingabevektoren abgebildet werden, mehr dazu in abschnitten 3.1.1 und 3.2.1 :)

\subsection{Regression vs Klassifikation}\label{section:regrvsclf}
Probleme aus dem Bereich des überwachten maschinellen Lernens lassen sich im Allgemeinen in eine von zwei Kategorien einordnen:
Klassifizierung bezeichnet den Prozess, bei dem ein Datensatz einer oder mehreren Klassen aus einer endlichen Liste möglicher Klassen zugeordnet wird. Dieser Ansatz findet beispielsweise bei der automatischen Kategorisierung von E-Mails (Spam oder nicht Spam) oder bei der Erkennung von handschriftlichen Texten (welches Symbol aus einem gegebenen Alphabet ist dargestellt?) Anwendung (CITE). 
Da es sich bei den betrachteten medizinischen Scores um diskrete, ganzzahlige Werte aus einem endlichen Wertebereich handelt, liegt auch hier die Anwendung eines Klassifizierungs-Verfahrens nahe.

Betrachtet man aber die verschiedenen möglichen Werte eines Scores als separate und voneinander unabhängige Klassen, so ginge eine wichtige Information über deren Anordnung verloren. Bei den im Rahmen dieser Arbeit behandelten Scores handelt es sich stets um eindimensionale, metrische Skalen. Im mathematischen Sinne stellen sie Totalordnungen dar: Sie erfüllen also die Anforderungen der Reflexivität, Antisymmetrie, Transitivität und Totalität. Bezeichne $M$ die Menge aller möglichen Werte eines beliebigen medizinischen Scores. Es gilt also für alle $a,b,c \in M$:

\begin{equation*}
    \centering
    \begin{aligned}[c]
        a \leq a\\
        a \leq b \land b \leq a \; \Rightarrow \; a=b\\
        a \leq b \land b \leq c \; \Rightarrow \; a \leq c\\
        a \leq b \lor b \leq a
    \end{aligned}
    \qquad
    \begin{aligned}[c]
        \text{(Reflexivität)}\\
        \text{(Antisymmetrie)}\\
        \text{(Transitivität)}\\
        \text{(Totalität)}
    \end{aligned}
\end{equation*}

Damit lassen sich die verschiedenen Scores vergleichen und in ein Verhältnis setzen. So ist ein RASS-Wert\footnote{Richmond Agitation-Sedation Scale} von $-4$ (tief sediert) beispielsweise deutlich näher an $-3$ (mäßig sediert) als an $+1$ (unruhig). Bei gängigen Verfahren zur Klassifizierung ginge diese Information verloren, da bei Kenngrößen zur Bewertung solcher Modelle nur betrachtet werden kann, ob ein gegebener Eingabetext genau der richtigen Kategorie (dem richtigen Score) zugeordnet wurde oder nicht. 

%(Darüber hinaus beruhen viele der Scores bei der Vergabe zumindest teilweise auch auf dem persönlichen Ermessen des behandelnden Arztes/der behandelnden Ärztin bzw. der Pflegekraft. Damit wäre selbst für menschliche Experten eine genaue Zuordnung eines Textes zu einer Punktzahl nicht immer möglich.)\footnote{checken ob das stimmt}
Bei der vorliegenden Arbeit habe ich mich demnach dafür entschieden, die Vergabe von Scores anhand von Eingabetexten als klassisches Regressionsproblem zu betrachten, und die Ausgaben der Modelle im Zweifelsfall auf den nächstmöglichen ganzzahligen Wert zu runden. Dieser Ansatz fand auch bei früheren Arbeiten, die sich mit ähnlichen Fragestellungen befassten, Anwendung (CITE). Die Abweichung des vorhergesagten Werts eines Modells von dem nächst möglichen Wert stellt somit sogar einen rudimentären Ansatz zur Bewertung der Konfidenz bei der Vergabe einzelner Werte dar.
%siehe: https://stats.stackexchange.com/a/282890

% \subsection{Natural Language Processing}
% Die Verarbeitung natürlicher Sprache stellt eine besondere Herausforderung im Gebiet des maschinellen Lernens dar.
%texte müssen featurized werden, weil ML Modelle nur Zahlen erwartet, keine Texte.

\subsection{Maschinelles Lernen in der Medizin}
Hier andere Einsatzgebiete von ML in Medizin \citep{krishnanSupervisedLearningApproach2018}. Und: Warum ML für unsere Problematik?

Problem, insbesondere in der Medizin: Explainable AI. Insbesondere bei ANN: Reihe von entscheidungen die von künstlichen neuronalen Netzwerken getroffen werden und zu einer bestimmten ausgabe führen sind für die Entwicker oft nicht nachvollziehbar (CITE). Das Teilgebiet XAI beschäftigt sich mit methoden, ML besser nachvollziehbar zu machen.
	
	\chapter{Übersicht über die Daten}
	\section{Datenerfassung an der Charité}
information über copra (fridtjof ausquetschen). bla bla bla

Die daten wurden vor dem Export pseudonymisiert. Nach dem Export lagen die Daten in Form mehrerer Datendateien vor. \texttt{patienten.csv} enthält Metainformationen über die betrachteten Patienten. Separat gibt \texttt{delir.csv} für jeden Patienten an, ob für diesen während seines Aufenthalts ein Delir diagnostiziert wurde.

\section{Übersicht über die vorliegenden Daten}

Der mir für diese Arbeit zur Verfügung stehende Datensatz enthält Informationen über insgesamt 1357 Patienten-Aufenthalte auf den Intensivstationen der Charité Berlin, die allesamt im Zeitraum von Juni 2019 bis Juni 2020 stattfanden. Jeder Aufenthalt wird über eine numerische, inkrementell aufsteigende Nummer\footnote{in den Datensätzen als n\_ID bezeichnet} eindeutig identifiziert. Hierbei gilt es zu beachten, dass ein Patient bzw. eine Patientin, der/die im Laufe seiner/ihrer Behandlung mehrere Male auf eine Intensivstation verlegt wird, für jeden separaten Aufenthalt eine neue Identifikationsnummer erhält. Für jeden Patient/für jede Patientin liegen Informationen über das Geschlecht, BMI\footnote{Body-Mass-Index}, das Alter zum Zeitpunkt der Aufnahme und ob während des Aufenthalts ein Delir\footnote{F05.* gemäß ICD-10} diagnostiziert wurde, vor. Weiterhin ist die Dauer des Aufenthalts auf der Intensivstation vermerkt, sowie ob der Patient/die Patientin während seines/ihres Aufenthalts verstorben ist. Die Mortalität während des Aufenthalts lag bei etwa 17\% ($n=225$). Die beobachteten Aufenthalte verliefen über Zeiträume von wenigen Stunden bis zu mehreren Monaten. Die mittlere Aufenthaltsdauer während des Beobachtungszeitraums liegt bei etwa 7,2 Tagen, der Median beträgt 3,2 Tage. Fast die Hälfte aller Aufenthalte endete also nach weniger als drei Tagen. Nur etwas mehr als ein Viertel der Patienten ($n=375$) wurden für eine Woche oder länger auf der Intensivstation behandelt.

Für jeden Patient/jede Patientin werden während der Dauer seines/ihres Aufenthalts medizinische Scores erfasst sowie Visitentexte geschrieben. Die Visitentexte werden digital erfasst und liegen im Unicode-Zeichensatz vor, folgen allerdings im Allgemeinen keinem einheitlichen Muster. Es handelt sich also um Freitexte, und es liegt im Ermessen der behandelnden Ärzte bzw. Pflegekräfte, einen aussagekräftigen Text zu formulieren. Ebenso können sich Faktoren wie Zeitdruck oder Stress auf Umfang und Genauigkeit der eingetragenen Texte auswirken(CITE).

%Diese Information über med. scores ggf. lieber in die Einleitung!
Bei den erfassten Scores handelt es sich um medizinische Bewertungssysteme, bei denen die Verfassung des betrachteten Patienten anhand klar definierter Regeln anhand einer Punktzahl bewertet wird(CITE). Häufig beziehen sich die Scores dabei nur auf einen kleinen Teil der Gesamtverfassung des Patienten, beispielsweise auf den Grad der Sedierung oder auf Schmerzempfinden, sofern dieses nicht vom Patienten selber mitgeteilt werden kann. Tabelle \ref{table:varids} gibt einen Übersicht über alle Scores und Freitexte, die in dem gegebenen Datensatz erfasst wurden. Bei der VarID handelt es sich um eine Zahl, mit der jede Art von auf der Intensivstation erfasstem Wert bzw. eingetragenem Text repräsentiert und eindeutig identifiziert wird.\footnote{Fridtjof fragen ob das stimmt}

Es folgt eine detailliertere Beschreibung derjenigen Werte, die für den Inhalt dieser Arbeit besonders hohe Relevanz haben:

\paragraph{Glasgow Coma Scale}
bla bla lba

%tabelle scores: varid, name, beschreibung, wertebereich
\begin{table}[htb]
    \centering
    \begin{tabular}{llr}\toprule
        \textbf{VarID}	&\textbf{Name} &\textbf{Wertebereich} \\\midrule
        20512769    & Glasgow Coma Scale (GCS)          & $3 \leq v \leq ?$ \\
        20512801    & Behavior Pain Scale (BPS)         & $3 \leq v \leq 12$ \\
        20512802    & Delirium Detection Score (DDS)    & $3 \leq v \leq 35$* \\
        22085815    & Visite\_ZNS                        & Freitext \\
        22085820    & Visite\_Oberarzt                   & Freitext \\
        22085836    & Visite\_Pflege                     & Freitext \\
        22085897    & Ramsay Sedation Scale             & $1 \leq v \leq 6$ \\
        22085911    & NRS/VAS (Visual Analogue Scale)   & $0 \leq v \leq 10$ \\
        22086067    & Vigilanz                          & Freitext* \\
        22086158    & Richmond Agitation Sedation Scale (RASS) & $-5 \leq v \leq 4$ \\
        22086169    & CAM-ICU                           & $v \in \{	\text{neg.}, \text{pos.}, \text{unmögl.} \}$ \\
        22086170    & BPS-Bewertung                     & Freitext* \\
        22086172    & NRS/VAS Bedingungen               & Freitext* \\\bottomrule
    \end{tabular}

    \caption{Übersicht aller erfassten Scores und Freitexte}
    \label{table:varids}
\end{table}
%hier jeweils ein kurzer Überblick über die wichtigsten werte.

%hier histogramme mit den erfassten werten der einzelnen scores.

\subsection{Exemplarische Vorstellung eines Patienten} %z.b. 0430
Aufenthalt des Patienten hier detailliert beschreiben, und seinen Scatterplot einfügen (aber noch ohne Pfeile).

\subsection{Generierung von Schlüssel-Wert-Paaren}
Eine besondere Herausforderung stellte dar, dass die verschiedenen Werte und Texte zu unterschiedlichen Zeiten und unabhängig voneinadner eingetragen werden. Die Informationen über Zeitpunkt und Art der Eintragung sowie der eigentliche Wert liegen jeweils als Tripel in Form mehrerer Log-Dateien vor. Bei den Visitentexten entspricht der eingetragene Text dem Wert der Eintragung.

Um ein Machine-Learning Modell aus dem Bereich des supervised learnings zu trainieren ist eine hohe Anzahl von Trainingspaaren notwendig. Jedes Wertepaar enthält einen Text sowie einen medizinischen Score, der den in dem Text angegebenen Informationen über die Verfasssung des Patient/die Patientin möglichst genau entspricht. Zusammen bilden diese Paare die Grundlage der Modelle, selbstständig noch nicht gesehene Texte bewerten zu können.
Aufgrund der zeitlichen Unterschiede zwischen den Eintragungen von Texten und Scores erwies sich allerdings ebenjene Zuordnung zueinander als nicht trivial.

% Hier Bild mit 4 Patienten-Scatterplots! Bildbeschreibung: "Patient X ist der Patient, dessen Aufenthalt in Sektion 2.2.2 detailliert vorgestellt wurde."

\subsection{Genauigkeit der erfassten Daten}
Hier beschreiben, dass viele Texte nicht wirklich den Werten entsprechen. Das mache es schwerer, Modelle zu trainieren, und muss berücksichtigt werden. Gründe:
\begin{enumerate}
    \item Zeitdruck, Stress bei Ärzten, kann vorkommen dass sie einfach Wert vom letzten mal kopieren
    \item Werte können sich innerhalb von Minuten ändern (z.B. RASS von 0 auf -5)
    \item Weitere Gründe?
\end{enumerate}
Diese Grüunde sind aber nicht weiter Beobachtungsgegenstand der vorleigenden Arbeit. Dennoch muss das bei der Konzeption, Entwicklung und Bewertung beachtet werden, weil sie ohne weitere Maßnahmen möglicherweise eine obere Schranke für die Performance der Modelle darstellen.

%   * **Vergleich der verschiedenen Methoden, Wertepaare aus den Daten zu generieren**
%     * nearest value pairs
%     * last text observation carried forward
%     * Vergleich anhand der scatter plots f. **einen** geeigneten Patienten
%   * Abb: Liniendiagramm Anzahl key-werte paare pro predicted varid sowie max cutoff time
	
	\chapter{Vorgehensweise}
	% bla bla bla

% * numerische optimierung: Lösungsraum, Bewertungsfunktion
%   * das Gradientenverfahren
%   * stochastic gradient descend
% * Evaluation von ML-Modellen
%       * z.b. cross validation
%       * vermeiden: data leakage
	
	\chapter{Auswertung der Ergebnisse}
	\dots

\section{Bewertung der Leitlinienadhärenz auf den Intensivstationen der Charité}
\dots

\section{Blick in die Zukunft}
\dots

\chapter{Fazit}
Brilliant formuliert, besonders die Konklusio!


% ## 4) Fazit
%   * Ergebnisse
%   * TODO: Auswertung der Visitentexte: Wo gibt es große Abweichungen von eingetragenem Score/Modell-Vorhersage?
%   * **Anwendung in der Zukunft**
%     * Vorgänge bei Datenerfassung auf ICU optimieren
%     * Erfassung von Qualität der bisher eingetragenen Daten
%     * z.b. einbeziehen der scores in clinicial decision support systems

	\newpage
	\bibliography{literatur}

	% Selbstständigkeitserklärung
	\includepdf[pages={1}]{Deckblatt/selbstaendigkeitserklaerung.pdf}
\end{document}