\documentclass[
	abstract=true, %abstract on!
	a4paper, % Papierformat A4
	12pt, % Schriftgröße 12pt
	bibliography=totoc, %bibliographie im Inhaltsverzeichnis
	headinclude, %scheint keinen Unterschied zu machen
	headsepline,
	draft, % replace me with final or comment out
	% Diese Optionen für digitale Version auskommentieren:
	twoside,
	BCOR=1mm, % 1mm sollte reichen für aufgeschlagenes Buch in Klebebindung
	open=right %Kapitelanfang immer auf rechter Seite, ggf. für pdf-Version weglassen
]{scrreprt}

\usepackage{pdfpages}

% \usepackage[titletoc]{appendix}
\usepackage{float}
\usepackage{caption}
\usepackage[]{subfig}
\usepackage{graphicx}
\graphicspath{ {fig/} }
\usepackage{xcolor}
\usepackage[english, main=ngerman]{babel} % englisch für Zitate auf Englisch
\usepackage[onehalfspacing]{setspace}
\usepackage[utf8]{inputenc}
\usepackage{microtype} % better kerning, word spacing etc
% \usepackage[colorlinks=true,allcolors=blue]{hyperref} %change options to [hidelinks] for printing!!

\usepackage{tikz}
\usetikzlibrary{shapes,arrows,positioning}
\tikzset{
block/.style = {rectangle, draw, text width=11em, text centered},
pipelinetext/.style = {rectangle, draw, text width=24em, text centered},
line/.style = {draw, -latex'},
}

\usepackage[hidelinks]{hyperref}
\usepackage{booktabs} % schönere Tabellen
\usepackage{xparse} % needed for \NewDocumentCommand
\usepackage[htt]{hyphenat} % hyphenate teletype text, z.b. "patienten.csv"
\usepackage{csquotes} % " wird zu Anführungszeichen
\MakeOuterQuote{"}

\newenvironment{itquote} % italics quote and " "
{\begin{quote}\itshape}
{\end{quote}}

% Rotation: \rot[<angle>][<width>]{<stuff>}
\NewDocumentCommand{\rot}{O{45} O{1em} m}{\makebox[#2][l]{\rotatebox{#1}{#3}}}%

\usepackage{amsmath} %z.b align
\usepackage{interval} %für Wertebereiche

\usepackage{footnote} %um fußnoten auch in Tabellen zu ermöglichen
\makesavenoteenv{tabular}
\makesavenoteenv{table}

\usepackage[square]{natbib} % scheinbar die beste Wahl für mich https://tex.stackexchange.com/a/25702/217229
\bibliographystyle{bibstyle.bst} % GUI to make bst style: latex makebst

% \usepackage{fontspec} % custom font (needs compiling via xelatex)
% \setmainfont{Fira Sans}

\newcommand{\refsec}[1]{{(siehe Abschnitt \ref{#1})}} %Referenz in Klammern auf Section
\newcommand{\reftbl}[1]{{(siehe Tabelle \ref{#1})}} %Referenz in Klammern auf Tabelle
\newcommand{\reffig}[1]{{(siehe Abbildung \ref{#1})}} %Referenz in Klammern auf Abbildung

\RedeclareSectionCommand[beforeskip=0pt,afterindent=false]{chapter} %weniger Abstand nach oben bei Kapitelüberschriften

% \usepackage{fontspec}
% \setmainfont{SourceSerifPro}[
%   Extension=.otf,
%   Path=./fonts/,
%   UprightFont=*-Regular,
%   BoldFont=*-Bold,
%   ItalicFont=*-It,
%   BoldItalicFont=*-BoldIt
% ]
% \setsansfont{FiraSans}[
%   Extension=.otf,
%   Path=./fonts/,
%   UprightFont=*-Regular,
%   BoldFont=*-Bold,
%   ItalicFont=*-Italic,
%   BoldItalicFont=*-BoldItalic,
%   FontFace={sb}{n}{FiraSans-SemiBold}
% ]

\begin{document}
	\pagestyle{empty}

	\renewcommand{\arraystretch}{1.15} %Zeilenabstand innerhalb von Tabellen
	% ========================================== %
	% Am Schluss noch überprüfen:
	% - einheitliche Zeitform
	% - Rechtschreibung
	% - Fußnoten alle groß, außer nur 1-2 Wörter
	% ========================================== %


	% Deckblatt:
	\includepdf[pages={1}]{Deckblatt/deckblatt.pdf}

	% Abstract
	\begin{abstract} \thispagestyle{plain} % Abstract soll auch Seitenzahl haben
	Die Verlegung eines Patienten auf eine Intensivstation erfolgt häufig infolge einer besonders schweren oder lebensbedrohlichen Erkrankung oder Verletzung. Insbesondere die digitale Kommunikation über den Gesundheitszustand der Patienten ist eine wichtige Vorraussetzung für ein optimales Therapieoutcome. Dennoch kommt es häufig zu unvollständigen, ungenauen oder widersprüchlichen Eintragungen. Das Ziel der vorliegenden Arbeit ist es, mithilfe von modernen Ansätzen des maschinellen Lernens medizinische Scores anhand von unstrukturierten Visitentexten vorherzusagen, um solche Missstände zu quantifizieren und zu beheben.
	\end{abstract}

	\pagestyle{headings} % Oben auf der Seite aktuelles Kapitel hinschreiben
	% Inhaltsverzeichnis:
	%\setlength{\parskip}{-em} % falls das Inhaltsverzeichnis eng wird
	\tableofcontents

	\setlength{\parskip}{.25em} % space between paragraphs. Erst nach dem Inhaltsverzeichnis:)

	\chapter{Einführung}
	\section{Datenerfassung auf Intensivstationen}
%   * worum geht es?
%   * Thematik
%     * Problematik der Datenerfassung auf ICUs

\cite{marxIntensivmedizin2015c} verorten die Intensivmedizin im Spannungsfeld zwischen Heilen und Sterben. Eine Verlegung auf die Intensivstation erfolgt häufig infolge einer besonders schweren oder lebensbedrohlichen Erkrankung oder Verletzung. Eine Behandlung auf der Intensivstation verfolgt, sofern möglich, das Ziel, den Patienten insofern zu kurieren, dass diesem ein Weiterleben unabhängig von den besonderen technischen und personellen Möglichkeiten der Intensivmedizin möglich ist. Dafür wird eine besonders intensive Behandlung durch Ärzte und Pflegekräfte benötigt.

Jüngere technische Entwicklungen ermöglichen es, mehr Informationen über jeden Patienten zu erfassen und zu verarbeiten als je zuvor. Ärzte, Pflegekräfte, aber auch Angehörige werden so mit großen Mengen an Informationen konfrontiert. Die erfolgreiche Kommunikation zwischen Behandelnden stellt eine wichtige Vorraussetzung für das Patientenoutcome dar. \cite{marxIntensivmedizin2015c} bezeichnen Kommunikationsprobleme als wichtigen Faktor für erhöhte Krankenhausmortalitätsraten. Die Autoren beschreiben weiter, dass bis zu 50 \% der klinisch relevanten Informationen, die noch in der Morgenvisite zwischen Ärzten ausgetauscht werden, schon in der Spätvisite des gleichen Tages nicht mehr übermittelt werden. %(Kap. 11.5.4)

Die Frage der effektiven Datenerfassung auf Intensivstationen ist also eine über Leben und Tod. Dennoch kommt es aus unterschiedlichen Gründen vor, dass Informationen über den Gesundheitszustand der Patienten ungenau oder in zu geringem Umfang digital erfasst werden. 

Gegenstand der vorliegenden Arbeit ist ein Versuch, mittels maschinellem Lernen einen Beitrag zur Lösung dieses Problems beizutragen.

\subsection{medizinische Scores} \label{section:scores}

In \textit{Die Intensivmedizin} \citep{marxIntensivmedizin2015c} ist der Begriff des Scores folgendermaßen definiert:

\begin{itquote}
    "Ein Score ist der Versuch, eine komplexe klinische Situation auf einen eindimensionalen Punktwert abzubilden. Eine solche Reduktion verfolgt das Ziel, übergreifende Aspekte wie Schweregrad oder Prognose als Kombination einzelner Fakten objektiv zu fassen, um sie dann in unterschiedlichen Kollektiven vergleichend darstellen zu können."
\end{itquote}

Es handelt sich bei einem Score häufig um die Kombination mehrerer erfassbarer Werte, beispielsweise der Herzfrequenz oder dem Sauerstoffgehalt im Blut. Auch allgemeine Informationen über den Patienten\footnote{Hier und im Rest der Arbeit umfasst das generische Maskulinum stets Personen beider Geschlechter.} wie das Alter oder bekannte Vorerkrankungen können berücksichtigt werden. Die Bestimmung eines Scores stellt also den Versuch dar, die komplexe, individuelle Situation eines Patienten auf einen numerischen Wert zu reduzieren. Dabei gehen unweigerlich Informationen verloren. Gleichzeitig erlaubt es die Erfassung von derartigen standardisierten Scores aber, auf einen Blick wichtige Informationen über den Zustand des Patienten zu erfassen. Durch eine derartige Reduktion auf das Wesentliche wird ferner ermöglicht, den pathologischen Verlauf eines Patienten über einen längeren Zeitraum zu analysieren, oder die Symptomatik mehrerer Patienten leichter miteinander zu vergleichen. Ein weiterer Vorteil ist es, dass, unter Voraussetzung der richtigen Anwendung, die Vergabe von Scores weitestgehend unabhängig von der subjektiven Einschätzung des Arztes oder der Pflegekraft erfolgt \citep{marxIntensivmedizin2015c}.
Eine Ausnahme unter den in den Datensätzen erfassten Werte bildet der CAM-ICU (siehe Abschnitt \ref{section:vorliegende_daten}). Das Ergebnis fällt hierbei entweder positiv oder negativ aus und stellt damit keinen Score im eigentlichen Sinne dar.
%Die Frage, ob es sich bei der Vorhersage der im Rahmen dieser Arbeit behandelten Scores um ein Regressions- oder ein Klassifikationsproblem handelt, wird in Abschnitt \ref{section:regrvsclf} weiter vertieft. 

\subsection{Ziel der Arbeit}
%Formulierung "Gegenstand der Arbeit" irgendwo verwenden
Das Ziel der vorliegenden Arbeit ist es, mit Hilfe von maschinellem Lernen ein statistisches Modell zu entwickeln, um anhand von Freitexten medizinische Scores möglichst akkurat vorherzusagen. Die Entwicklung eines solchen Modells ermöglicht es unter anderem, die tatsächlich eingetragenen Werte mit den Vorhersagen des Modells zu vergleichen, um daraus Rückschlüsse über die Qualtät der Datenerfassung an der Charité zu treffen.

\section{Maschinelles Lernen}
% Hier auf historischen Kontext eingehen! (big data, processing power Moore's law etc)

Der Begriff Maschinelles Lernen bezeichnet einen modernen Ansatz in der Arbeit an künstlicher Intelligenz. (cite).

hat sich im Zeitalter von Big Data und leistungsstarken Rechnern zu einem der Hauptforschungspunkte der Informatik entwicklt (cite). 

\citet{mitchellMachineLearning1997} definiert den Vorgang maschinellen Lernens folgendermaßen:

\begin{itquote}
    {\foreignlanguage{english}{"A computer program is said to learn from experience E with respect to some class of tasks T and performance measure P, if its performance at tasks in T, as measured by P, improves with experience E."}}
\end{itquote}

%In der vorliegenden Arbeit handelt es sich bei E um bereits gelabelte Wertepaare und bei T um die Aufgabe, einem noch unbekannten Text den richtigen Wert aus dem Wertebereich eines medizinischen Scores zuzuordnen. Für P kommen verschiedene Metriken in Frage, um die Qualität des Modells zu bemessen, beispielsweise das Bestimmtheitsmaß R^2 oder MAE (siehe Abschnitt 1.2.3 performance metrics :))

Im Allgemeinen werden also Algorithmen, die aus großen Datensätzen "lernen" und damit Vorhersagen über unbekannte Daten machen, als maschinelles Lernen bezeichnet.

\subsection{Überwachtes Lernen}\label{section:supervised_learning}
Beschreiben: Hier gehts um supervised ML! (neben supervised gibt es noch: unsupervised, wo foo, und Reinforcement learning, wo bar.)

Texte müssen in numerische Eingabevektoren abgebildet werden, mehr dazu in abschnitten 3.1.1 und 3.2.1 :)

\subsection{Regression vs Klassifikation}\label{section:regrvsclf}
Probleme aus dem Bereich des überwachten maschinellen Lernens lassen sich im Allgemeinen in eine von zwei Kategorien einordnen:
Klassifizierung bezeichnet den Prozess, bei dem ein Datensatz einer oder mehreren Klassen aus einer endlichen Liste möglicher Klassen zugeordnet wird. Dieser Ansatz findet beispielsweise bei der automatischen Kategorisierung von E-Mails (Spam oder nicht Spam) oder bei der Erkennung von handschriftlichen Texten (welches Symbol aus einem gegebenen Alphabet ist dargestellt?) Anwendung (CITE). 
Da es sich bei den betrachteten medizinischen Scores um diskrete, ganzzahlige Werte aus einem endlichen Wertebereich handelt, liegt auch hier die Anwendung eines Klassifizierungs-Verfahrens nahe.

Betrachtet man aber die verschiedenen möglichen Werte eines Scores als separate und voneinander unabhängige Klassen, so ginge eine wichtige Information über deren Anordnung verloren. Bei den im Rahmen dieser Arbeit behandelten Scores handelt es sich stets um eindimensionale, metrische Skalen. Im mathematischen Sinne stellen sie Totalordnungen dar: Sie erfüllen also die Anforderungen der Reflexivität, Antisymmetrie, Transitivität und Totalität. Bezeichne $M$ die Menge aller möglichen Werte eines beliebigen medizinischen Scores. Es gilt also für alle $a,b,c \in M$:

\begin{equation*}
    \centering
    \begin{aligned}[c]
        a \leq a\\
        a \leq b \land b \leq a \; \Rightarrow \; a=b\\
        a \leq b \land b \leq c \; \Rightarrow \; a \leq c\\
        a \leq b \lor b \leq a
    \end{aligned}
    \qquad
    \begin{aligned}[c]
        \text{(Reflexivität)}\\
        \text{(Antisymmetrie)}\\
        \text{(Transitivität)}\\
        \text{(Totalität)}
    \end{aligned}
\end{equation*}

Damit lassen sich die verschiedenen Scores vergleichen und in ein Verhältnis setzen. So ist ein RASS-Wert\footnote{Richmond Agitation-Sedation Scale} von $-4$ (tief sediert) beispielsweise deutlich näher an $-3$ (mäßig sediert) als an $+1$ (unruhig). Bei gängigen Verfahren zur Klassifizierung ginge diese Information verloren, da bei Kenngrößen zur Bewertung solcher Modelle nur betrachtet werden kann, ob ein gegebener Eingabetext genau der richtigen Kategorie (dem richtigen Score) zugeordnet wurde oder nicht. 

%(Darüber hinaus beruhen viele der Scores bei der Vergabe zumindest teilweise auch auf dem persönlichen Ermessen des behandelnden Arztes/der behandelnden Ärztin bzw. der Pflegekraft. Damit wäre selbst für menschliche Experten eine genaue Zuordnung eines Textes zu einer Punktzahl nicht immer möglich.)\footnote{checken ob das stimmt}
Bei der vorliegenden Arbeit habe ich mich demnach dafür entschieden, die Vergabe von Scores anhand von Eingabetexten als klassisches Regressionsproblem zu betrachten, und die Ausgaben der Modelle im Zweifelsfall auf den nächstmöglichen ganzzahligen Wert zu runden. Dieser Ansatz fand auch bei früheren Arbeiten, die sich mit ähnlichen Fragestellungen befassten, Anwendung (CITE). Die Abweichung des vorhergesagten Werts eines Modells von dem nächst möglichen Wert stellt somit sogar einen rudimentären Ansatz zur Bewertung der Konfidenz bei der Vergabe einzelner Werte dar.
%siehe: https://stats.stackexchange.com/a/282890

% \subsection{Natural Language Processing}
% Die Verarbeitung natürlicher Sprache stellt eine besondere Herausforderung im Gebiet des maschinellen Lernens dar.
%texte müssen featurized werden, weil ML Modelle nur Zahlen erwartet, keine Texte.

\subsection{Maschinelles Lernen in der Medizin}
Hier andere Einsatzgebiete von ML in Medizin \citep{krishnanSupervisedLearningApproach2018}. Und: Warum ML für unsere Problematik?

Problem, insbesondere in der Medizin: Explainable AI. Insbesondere bei ANN: Reihe von entscheidungen die von künstlichen neuronalen Netzwerken getroffen werden und zu einer bestimmten ausgabe führen sind für die Entwicker oft nicht nachvollziehbar (CITE). Das Teilgebiet XAI beschäftigt sich mit methoden, ML besser nachvollziehbar zu machen.
	
	\chapter{Übersicht über die Daten} % DONE %
	\section{Datenerfassung an der Charité}
information über copra (fridtjof ausquetschen). bla bla bla

Die daten wurden vor dem Export pseudonymisiert. Nach dem Export lagen die Daten in Form mehrerer Datendateien vor. \texttt{patienten.csv} enthält Metainformationen über die betrachteten Patienten. Separat gibt \texttt{delir.csv} für jeden Patienten an, ob für diesen während seines Aufenthalts ein Delir diagnostiziert wurde.

\section{Übersicht über die vorliegenden Daten}

Der mir für diese Arbeit zur Verfügung stehende Datensatz enthält Informationen über insgesamt 1357 Patienten-Aufenthalte auf den Intensivstationen der Charité Berlin, die allesamt im Zeitraum von Juni 2019 bis Juni 2020 stattfanden. Jeder Aufenthalt wird über eine numerische, inkrementell aufsteigende Nummer\footnote{in den Datensätzen als n\_ID bezeichnet} eindeutig identifiziert. Hierbei gilt es zu beachten, dass ein Patient bzw. eine Patientin, der/die im Laufe seiner/ihrer Behandlung mehrere Male auf eine Intensivstation verlegt wird, für jeden separaten Aufenthalt eine neue Identifikationsnummer erhält. Für jeden Patient/für jede Patientin liegen Informationen über das Geschlecht, BMI\footnote{Body-Mass-Index}, das Alter zum Zeitpunkt der Aufnahme und ob während des Aufenthalts ein Delir\footnote{F05.* gemäß ICD-10} diagnostiziert wurde, vor. Weiterhin ist die Dauer des Aufenthalts auf der Intensivstation vermerkt, sowie ob der Patient/die Patientin während seines/ihres Aufenthalts verstorben ist. Die Mortalität während des Aufenthalts lag bei etwa 17\% ($n=225$). Die beobachteten Aufenthalte verliefen über Zeiträume von wenigen Stunden bis zu mehreren Monaten. Die mittlere Aufenthaltsdauer während des Beobachtungszeitraums liegt bei etwa 7,2 Tagen, der Median beträgt 3,2 Tage. Fast die Hälfte aller Aufenthalte endete also nach weniger als drei Tagen. Nur etwas mehr als ein Viertel der Patienten ($n=375$) wurden für eine Woche oder länger auf der Intensivstation behandelt.

Für jeden Patient/jede Patientin werden während der Dauer seines/ihres Aufenthalts medizinische Scores erfasst sowie Visitentexte geschrieben. Die Visitentexte werden digital erfasst und liegen im Unicode-Zeichensatz vor, folgen allerdings im Allgemeinen keinem einheitlichen Muster. Es handelt sich also um Freitexte, und es liegt im Ermessen der behandelnden Ärzte bzw. Pflegekräfte, einen aussagekräftigen Text zu formulieren. Ebenso können sich Faktoren wie Zeitdruck oder Stress auf Umfang und Genauigkeit der eingetragenen Texte auswirken(CITE).

%Diese Information über med. scores ggf. lieber in die Einleitung!
Bei den erfassten Scores handelt es sich um medizinische Bewertungssysteme, bei denen die Verfassung des betrachteten Patienten anhand klar definierter Regeln anhand einer Punktzahl bewertet wird(CITE). Häufig beziehen sich die Scores dabei nur auf einen kleinen Teil der Gesamtverfassung des Patienten, beispielsweise auf den Grad der Sedierung oder auf Schmerzempfinden, sofern dieses nicht vom Patienten selber mitgeteilt werden kann. Tabelle \ref{table:varids} gibt einen Übersicht über alle Scores und Freitexte, die in dem gegebenen Datensatz erfasst wurden. Bei der VarID handelt es sich um eine Zahl, mit der jede Art von auf der Intensivstation erfasstem Wert bzw. eingetragenem Text repräsentiert und eindeutig identifiziert wird.\footnote{Fridtjof fragen ob das stimmt}

Es folgt eine detailliertere Beschreibung derjenigen Werte, die für den Inhalt dieser Arbeit besonders hohe Relevanz haben:

\paragraph{Glasgow Coma Scale}
bla bla lba

%tabelle scores: varid, name, beschreibung, wertebereich
\begin{table}[htb]
    \centering
    \begin{tabular}{llr}\toprule
        \textbf{VarID}	&\textbf{Name} &\textbf{Wertebereich} \\\midrule
        20512769    & Glasgow Coma Scale (GCS)          & $3 \leq v \leq ?$ \\
        20512801    & Behavior Pain Scale (BPS)         & $3 \leq v \leq 12$ \\
        20512802    & Delirium Detection Score (DDS)    & $3 \leq v \leq 35$* \\
        22085815    & Visite\_ZNS                        & Freitext \\
        22085820    & Visite\_Oberarzt                   & Freitext \\
        22085836    & Visite\_Pflege                     & Freitext \\
        22085897    & Ramsay Sedation Scale             & $1 \leq v \leq 6$ \\
        22085911    & NRS/VAS (Visual Analogue Scale)   & $0 \leq v \leq 10$ \\
        22086067    & Vigilanz                          & Freitext* \\
        22086158    & Richmond Agitation Sedation Scale (RASS) & $-5 \leq v \leq 4$ \\
        22086169    & CAM-ICU                           & $v \in \{	\text{neg.}, \text{pos.}, \text{unmögl.} \}$ \\
        22086170    & BPS-Bewertung                     & Freitext* \\
        22086172    & NRS/VAS Bedingungen               & Freitext* \\\bottomrule
    \end{tabular}

    \caption{Übersicht aller erfassten Scores und Freitexte}
    \label{table:varids}
\end{table}
%hier jeweils ein kurzer Überblick über die wichtigsten werte.

%hier histogramme mit den erfassten werten der einzelnen scores.

\subsection{Exemplarische Vorstellung eines Patienten} %z.b. 0430
Aufenthalt des Patienten hier detailliert beschreiben, und seinen Scatterplot einfügen (aber noch ohne Pfeile).

\subsection{Generierung von Schlüssel-Wert-Paaren}
Eine besondere Herausforderung stellte dar, dass die verschiedenen Werte und Texte zu unterschiedlichen Zeiten und unabhängig voneinadner eingetragen werden. Die Informationen über Zeitpunkt und Art der Eintragung sowie der eigentliche Wert liegen jeweils als Tripel in Form mehrerer Log-Dateien vor. Bei den Visitentexten entspricht der eingetragene Text dem Wert der Eintragung.

Um ein Machine-Learning Modell aus dem Bereich des supervised learnings zu trainieren ist eine hohe Anzahl von Trainingspaaren notwendig. Jedes Wertepaar enthält einen Text sowie einen medizinischen Score, der den in dem Text angegebenen Informationen über die Verfasssung des Patient/die Patientin möglichst genau entspricht. Zusammen bilden diese Paare die Grundlage der Modelle, selbstständig noch nicht gesehene Texte bewerten zu können.
Aufgrund der zeitlichen Unterschiede zwischen den Eintragungen von Texten und Scores erwies sich allerdings ebenjene Zuordnung zueinander als nicht trivial.

% Hier Bild mit 4 Patienten-Scatterplots! Bildbeschreibung: "Patient X ist der Patient, dessen Aufenthalt in Sektion 2.2.2 detailliert vorgestellt wurde."

\subsection{Genauigkeit der erfassten Daten}
Hier beschreiben, dass viele Texte nicht wirklich den Werten entsprechen. Das mache es schwerer, Modelle zu trainieren, und muss berücksichtigt werden. Gründe:
\begin{enumerate}
    \item Zeitdruck, Stress bei Ärzten, kann vorkommen dass sie einfach Wert vom letzten mal kopieren
    \item Werte können sich innerhalb von Minuten ändern (z.B. RASS von 0 auf -5)
    \item Weitere Gründe?
\end{enumerate}
Diese Grüunde sind aber nicht weiter Beobachtungsgegenstand der vorleigenden Arbeit. Dennoch muss das bei der Konzeption, Entwicklung und Bewertung beachtet werden, weil sie ohne weitere Maßnahmen möglicherweise eine obere Schranke für die Performance der Modelle darstellen.

%   * **Vergleich der verschiedenen Methoden, Wertepaare aus den Daten zu generieren**
%     * nearest value pairs
%     * last text observation carried forward
%     * Vergleich anhand der scatter plots f. **einen** geeigneten Patienten
%   * Abb: Liniendiagramm Anzahl key-werte paare pro predicted varid sowie max cutoff time
	
	\chapter{Baseline-Modell}\label{chap-vorgehensweise}
	% Für die Konzeption eines geeigneten Modells zur Vorhersage medizinischer Scores habe ich mich dafür entschieden, zwei separate, voneinander unabhängige Ansätze zu verfolgen. In diesem Kapitel wird die Entwicklung eines einfachen Regressionsmodells anhand einer sogenannten Support Vector Machine beschrieben. Kapitel \ref{chapter:ELM} beinhaltet die Entwicklung eines Modells basierend auf der Extreme Learning Machine, einem Verfahren, das es ermöglicht, künstliche neuronale Netze mit nur einem hidden layer effizient zu trainieren \citep{huangExtremeLearningMachine2006}. Weiterhin werden dort weitere Methoden zur Vorverarbeitung der Eingabedaten vorgestellt sowie deren Wirkung bewertet. 

\section{Datenaufbereitung}\label{sec:datenaufbereitung}

Die Verarbeitung natürlicher Sprache stellt eine besondere Herausforderung für das maschinelle Lernen dar. Im Allgemeinen sind Texte, also Aneinanderreihungen von Buchstaben und Symbolen variabler Länge, keine geeigneten Eingabedaten für die Modelle des überwachten maschinellen Lernens. Daher müssen sie zuerst in eine angemessene numerische Repräsentation überführt werden. Dabei ist es wichtig, dass der für die Vorhersage relevante Inhalt des Textes so gut wie möglich erhalten bleibt. Dieser Prozess, bei dem jeweils ein Eingabetext auf einen abstrakten Vektor reeller Zahlen abgebildet wird (den sog. \textit{feature vector}), wird als \textit{word embedding} bezeichnet. 

\subsection{Tokenisierung}

\subsubsection{Bag-of-Words}
Ein trivialer Ansatz ist das sogenannte Bag-of-Words-Modell. Hierbei wird zunächst jeder Eingabetext in eine Liste seiner \textit{tokens} umgewandelt (Tokenisierung). Im einfachsten Fall ist jedes Wort, getrennt durch eines oder mehrere Leerzeichen, ein solches Token:

\begin{figure}[H]
    \centering
    \begin{tikzpicture}[node distance = 5em, auto]
        % Place nodes
        \node [block] (in) {\texttt{'Der Patient schläft tief und fest.'}};
        \node [block, right = of in] (out) {\texttt{['Der', 'Patient', 'schläft', 'tief', 'und', 'fest.']}};
        % Draw edges
        \path [line] (in) -- (out);
    \end{tikzpicture}
    \caption{}
    \label{fig:tokenize_words}
\end{figure}

Anschließend wird jedem Wort, das in mindestens einem der Eingabetexte auftritt, eine feste Position in den feature-Vektoren zugeordnet. Die Länge der Vektoren entspricht somit der Anzahl einzigartiger Worte in allen Eingabetexten.
Zuletzt werden die Vorkommnisse der Worte in jedem Text gezählt und ihre Summe an der entsprechenden Stelle des dazugehörigen Vektors eingetragen:

\begin{table}[h]
\centering
\begin{tabular}{lcccccccccc}
    & \rot[90]{Der}
    & \rot[90]{Patient}
    & \rot[90]{schläft}
    & \rot[90]{tief}
    & \rot[90]{und}
    & \rot[90]{fest}
    & \rot[90]{ist}
    & \rot[90]{nicht}
    & \rot[90]{agitiert}
    & \rot[90]{sediert}\\
    \midrule
    Der Patient schläft tief und fest      & 1 & 1 & 1 & 1 & 1 & 1 & 0 & 0 & 0 & 0 \\
    Patient ist nicht agitiert und schläft & 0 & 1 & 1 & 0 & 1 & 0 & 1 & 1 & 1 & 0 \\
    Der Patient ist tief sediert           & 1 & 1 & 0 & 1 & 0 & 0 & 1 & 0 & 0 & 1 \\
    \bottomrule
\end{tabular}
\caption{Absolute Häufigkeit von Worten in drei Beispieltexten}
\end{table}

Die Eingabedaten bestehen somit aus einer Matrix, deren Zeilen die Eingabetexte und deren Spalten die Anzahl der Vorkommnisse eines bestimmten Wortes enthalten. 

\subsubsection{N-Gramme}
Ein Nachteil dieser Art der Repräsentation ist, dass die Information über die Reihenfolge der Wörter innerhalb eines Textes verloren geht: Zwei Texte mit den gleichen Worten in einer unterschiedlichen Reihenfolge werden somit auf den gleichen Vektor abgebildet. Dieses Problem wird durch die Einführung sogenannter Bigramme vermindert: Hierbei werden jeweils zwei aufeinanderfolgende Worte zusammen als ein Token aufgefasst:

\begin{center}\begin{tikzpicture}[node distance = 5em, auto]
    % Place nodes
    \node [block] (in) {\texttt{'Der Patient schläft tief und fest.'}};
    \node [block, right = of in] (out) {\texttt{['Der Patient', 'Patient schläft', 'schläft tief', 'tief und', 'und fest.']}};
    % Draw edges
    \path [line] (in) -- (out);
\end{tikzpicture}\end{center}

Das Bigramm ist eine spezielle Form des allgemeinen N-Gramms, bei dem n die Anzahl der aufeinanderfolgenden Wörter angibt, die zu einem Token zusammengesetzt werden. Die Verwendung von N-Grammen bei der Tokenisierung ermöglicht es somit, einige der semantischen Informationen des Ausgangstextes zu erhalten, die bei einem einfachen Bag-of-Words-Modell verloren gehen würden. Gleichzeitig wird die Dimensionalität der Eingabevektoren drastisch erhöht, was zu höheren Anforderungen an Rechenkapazität und Speicherplatz beim Trainieren des Modells führt. 

Statt auf Wortebene lässt sich die Tokenisierung der Eingabetexte auch auf Zeichenebene durchführen. Die Repräsentation des Textes wird somit deutlich granularer und höherdimensionierter, indem bei einem N-Gramm statt n Worten n aufeinanderfolgende Zeichen (d.h. Buchstaben oder Zahlen) zu einem Token zusammengesetzt werden. Bei einer Sonderform dieser sogenannten \textit{character n-grams} werden nur solche Zeichenfolgen von Länge n betrachtet, die Teil eines einzelnen Wortes sind und nicht über Wortgrenzen hinausgehen. Die Auswirkungen der Wahl von n sowie des Verfahrens zur Tokenisierung werden im Abschnitt GridSearchCV genauer erläutert.

\subsubsection{Tf-idf-Maß}
Einige Worte (z.B. "Patient", "Verlauf") treten in vielen Texten auf und haben dadurch eine geringere Bedeutung bei der Vorhersage eines Score-Werts.
Seltenere Worte hingegen, die eine hohe Bedeutung haben können, müssen demnach entsprechend gewichtet werden, um diesen bei der Verarbeitung durch das Modell einen höheren Einfluss auf die Ausgabe zu ermöglichen. 

Zu diesem Zweck wird das sogenannte Tf-idf-Maß ermittelt, welches dem Produkt der \textit{term frequency} und der \textit{inverse document frequency} entspricht. Der Begriff \textit{term frequency} bezeichnet hierbei lediglich die bereits berechnete absolute Häufigkeit eines Wortes (bzw. Tokens im Allgemeinen), d.h. die Summe seiner Vorkommnisse innerhalb eines Texts. Die \textit{inverse document frequency} gibt die Aussagekraft eines bestimmten Wortes an, indem seine relative Häufigkeit in allen Texten bestimmt wird. Sei $D$ die Menge aller Texte und $t$ ein Token, so ist $\mathrm{idf}(t, D)$ der logarithmisch skalierte Kehrwert des Anteils derjenigen Texte aus $D$, in denen $t$ mindestens ein mal vorkommt:

\[ \mathrm{idf}(t, D) =  \log \frac{|D|}{|\{d \in D: t \in d\}|} \]

Hierbei ist zu beachten, dass $\mathrm{idf}(t, D)$ nur für solche Tokens $t$ definiert ist, die in mindestens einem Text in $D$ auftreten, da sonst $\{d \in D: t \in d\} = \emptyset$ mit $|\emptyset| = 0$.

\begin{table}[h]
    \centering
    \begin{tabular}{lcccccccccc}
        & \rot[90]{Der}
        & \rot[90]{Patient}
        & \rot[90]{schläft}
        & \rot[90]{tief}
        & \rot[90]{und}
        & \rot[90]{fest}
        & \rot[90]{ist}
        & \rot[90]{nicht}
        & \rot[90]{agitiert}
        & \rot[90]{sediert}\\
        \midrule
        Der Patient schläft tief und fest      & \footnotesize{.18} & 0 & \footnotesize{.18} & \footnotesize{.18} & \footnotesize{.18} & \footnotesize{.48} & 0     & 0     & 0     & 0 \\
        Patient ist nicht agitiert und schläft & 0     & 0 & \footnotesize{.18} & 0     & \footnotesize{.18} & 0     & \footnotesize{.18} & \footnotesize{.48} & \footnotesize{.48} & 0 \\
        Der Patient ist tief sediert       & \footnotesize{.18} & 0 & 0     & \footnotesize{.18} & 0     & 0     & \footnotesize{.18} & 0     & 0     & \footnotesize{.48} \\
        \bottomrule
    \end{tabular}
\caption{Tf-idf-Maß von Worten in den gleichen Beispieltexten}
\label{tab:idftab}
\end{table}

In Tabelle \ref{tab:idftab} ist erkennbar, dass Worte, die in weniger Texten auftreten, einen höheren Tf-idf-Wert haben. "Patient" tritt in allen betrachteten Texten auf und hat einen Wert von 0, da $\mathrm{idf}(\mathrm{'Patient'}, D) = \log \frac{3}{3} = 0$. Somit ist dieses Wort, zumindest bei dem gegebenen Beispiel-Korpus, irrelevant.

In der verwendeten Implementierung von scikit-learn werden die $n$-dimensionalen Tf-idf-Vektoren der Eingabetexte anschließend normiert, indem sie durch ihre euklidische Norm dividiert werden:

\[v_{norm} = \frac{v}{||v||_2} = \frac{v}{\sqrt{
    \sum_{i=0}^n (v_i)^2
}}\]

\subsection{weitere Datenbereinigung}
Abbildung \ref{fig:tokenize_words} zeigt exemplarisch die Überführung eines Beispieltexts in seine Tokens und verdeutlicht ein weiteres häufiges Problem der Textverarbeitung: Die Eingabedaten enthalten Satz- und Sonderzeichen, die eine sinnvolle Tokenisierung weiter erschweren. Der Beispieltext endet, wie viele der realen Eingabedaten, mit einem Punkt. Somit wird "fest." als neues Token erkannt, welches aus Sicht des Modells komplett unabhängig zu dem womöglich ebenfalls auftretenden "fest" ist. Dies erhöht nicht nur die Dimensionalität der Eingabedaten unnötig, sondern erschwert es dem Modell auch, die Bedeutung solcher Worte bei der Vorhersage der medizinischen Scores zu ermitteln\footnote{Die Abkürzung "pat" für Patient tritt in den bereitgestellten Eingabedaten ebenfalls häufig (n = 37.103) und mit unterschiedlicher Großschreibung auf, wird aber in nur etwa 66.8 \% (n = 24.810) der Fälle mit einem Punkt ("pat.") geschrieben. Ohne weitere Schritte zur Datenbereinigung würden diese beiden Worte komplett separat behandelt werden, obwohl sie semantisch identisch sind.}.

Bevor ein Eingabetext für die Verarbeitung durch das Modell tokenisiert wird, wird er in vier Schritten sukzessive vereinfacht und auf seine semantischen Kerninhalte reduziert (siehe Abbildung \ref{fig:text_pipeline}). Ziel ist es, die Texte auf ihre inhaltlichen Kerninhalte zu reduzieren, sodass die verbleibenden Eingabedaten eine maximal hohe Aussagekraft bei der Bestimmung der medizinischen Scores haben.

Im ersten Schritt wird jedes großgeschriebene Satzzeichen durch den entsprechenden Kleinbuchstaben ersetzt. Somit spielt die Großschreibung bei der Unterscheidung der Tokens keine Rolle mehr. 
Danach werden sämtliche Sonderzeichen, d.h. solche, die nicht Teil des deutschen Alphabets und keine Zahl sind, aus dem Text entfernt, da diese ebenfalls keine Bedeutung bei der Bestimmung des passenden Score-Werts haben.

In Schritt drei wird jedes verbleibende Wort auf seinen Wortstamm zurückgeführt, da die verschiedenen syntaktischen Formen eines Wortes ebenfalls unerheblich sind. Hierbei findet eine Python-Implementierung des Stemming-Algorithmus\footnote{http://snowball.tartarus.org/algorithms/german/stemmer.html} von Dr. Martin Porter Anwendung. Dieser entfernt Suffixe deutscher Worte anhand einer Folge wohldefinierter Regeln und benötigt somit kein Wörterbuch, um die Worte auf ihre Wortstämme zu reduzieren.

Schlussendlich werden sogenannte Stoppwörter entfernt. Dies sind Wörter, die in der deutschen Sprache häufig vorkommen und somit nur eine syntaktische Bedeutung haben, bei der Deutung des Inhalts eines Textes aber unerheblich sind. Grundlage hierfür bildet eine Liste von 232 deutschen Stoppwörtern aus dem Korpus des Open-Source-Projekts NLTK\footnote{Natural Language Toolkit: https://www.nltk.org/}. Da die Stammformreduktion bereits im vorherigen Schritt stattfand, muss die Liste der Stoppwörter entsprechend angepasst werden, um im Text die Worte korrekt zu entfernen. Insgesamt zwölf Worte wie "nicht", "ohne" und "keine" wurden manuell aus der NLTK-Liste entfernt, da diese eine inhaltlich relevante Bedeutung in den Texten haben können. 

\begin{figure}[h]
    \centering
    \begin{tikzpicture}[node distance = 4em, auto]
        % Place nodes
        \node [pipelinetext]               (1) {\texttt{Pat ist koop. und adäquat, mobi ohne Probleme, ECMO ohne Probleme, Pat hat gegessen und getrunken}};
        \node [pipelinetext, below = of 1] (2) {\texttt{pat ist koop. und adäquat, mobi ohne probleme, ecmo ohne probleme, pat hat gegessen und getrunken}};
        \node [pipelinetext, below = of 2] (3) {\texttt{pat ist koop und adäquat mobi ohne probleme ecmo ohne probleme pat hat gegessen und getrunken}};
        \node [pipelinetext, below = of 3] (4) {\texttt{pat ist koop und adaquat mobi ohn problem ecmo ohn problem pat hat gegess und getrunk}};
        \node [pipelinetext, below = of 4] (5) {\texttt{pat koop adaquat mobi ohn problem ecmo ohn problem pat gegess getrunk}};
        
        % Draw edges
        \draw[thick,->] (1) -- (2) node[midway,right] {Kleinschreibung};
        \draw[thick,->] (2) -- (3) node[midway,right] {Sonderzeichen entfernen};
        \draw[thick,->] (3) -- (4) node[midway,right] {Stammformreduktion};
        \draw[thick,->] (4) -- (5) node[midway,right] {Stoppwörter entfernen};
    \end{tikzpicture}
    \caption{}
    \label{fig:text_pipeline}
\end{figure}

% Einige der Stammformreduktionen im letzten Schritt wirken auf den ersten Blick nicht zielführend, da sie den Text für menschliche Leser unter Umständen schwerer verständlich machen. 
% Worte wie gegessen haben für den computer keine inhernänte bedeutung, sondern erhalten ihre bedeutung durch die trainingspaare, die ebenfalls gestemmt sind. mag für uns komisch wirken, ist aber kein problem. genauere betrachtung der ergebnisse/performance im abschnitt gridsearchcv

\section{Support Vector Machine}
Die sogenannte \textit{Support Vector Machine} ist ein Modell des überwachten Lernens, das einen hohen Grad an Generalisierung ermöglicht, d.h. auch bei neuen Daten, die nicht zum Trainieren des Modells genutzt wurden, eine hohe Genauigkeit erbringen kann \citep{Awad2015}. Während das Konzept der SVM ursprünglich zur Lösung von Klassifizierungsproblemen erdacht wurde, bezeichnet \textit{Support Vector Regression} eine Modifikation des gleichen Konzepts zur Regression. Wie bei anderen Regressionsmodellen wird versucht, im $n$-dimensionalen Raum der Eingabedaten eine Hyperebene zu finden, die den Zusammenhang zwischen Ein- und Ausgabedaten möglichst genau modelliert und somit Vorhersagen über Eingaben (Visitentexte), deren dazugehörigen Ausgaben (Score-Werte) noch unbekannt sind, ermöglicht. Haben die Eingabedaten nur eine Dimension, entspricht dies anschaulich einer Regressionsgeraden, die durch die Punktwolke $\{(x_1, y_1), \dots, (x_n, y_n)\}$ der $n$ Wertepaare gelegt wird. Sind die Eingabedaten zweidimensionale Vektoren, so entspricht die Ausgabe des Modells einer Fläche im 3-dimensionalen Raum. Die in Abschnitt \ref{sec:datenaufbereitung} beschriebene Überführung der Eingabetexte in reelle Vektoren liefert deutlich höherdimensionierte Daten, die weniger anschaulich sind, bei denen das Konzept aber das gleiche bleibt.

Eine solche Hyperebene (bzw. Gerade im 2-dimensionalen Raum) kann nur gute Ergebnisse liefern, wenn eine lineare Abhängigkeit zwischen Ein- und Ausgabedaten besteht. Da dies im Allgemeinen nicht der Fall ist, wird bei der SVM der Kernel-Trick angewendet. Hierbei werden die Daten in  noch höher dimensionierte Räume überführt, bis eine annähernd lineare Trennbarkeit gegeben ist.

Reale Eingabedaten sind häufig ungenau und fehlerbelastet (siehe Abschnitt \ref{section:genauigkeit_der_daten}). Anhand dieser Daten ein Modell zu entwickeln, welches jedem möglichen neuen Visitentext den korrekten Score-Wert zuordnet ist somit unmöglich. Eine weitere Besonderheit der SVM liegt bei der Berechnung der Verlustfunktion in der Einführung der Variable $\epsilon$. Diese gibt eine maximal erlaubte Toleranz bei der Vorhersage der Ausgabedaten an. Liegt die Vorhersage des Modells weniger als $\epsilon$ von dem tatsächlichen Wert entfernt, so beträgt der Wert der Verlustfunktion für diesen Eingabewert 0. Die Einführung einer solchen Toleranz ermöglicht es, ein Modell zu trainieren, das weniger anfällig gegenüber widersprüchlichen Eingabedaten ist und somit eine bessere Generalisierung im Allgemeinen ermöglicht \citep{Awad2015}.

\subsection{Suche nach den besten Parametern}\label{sec:hyperparams}
Für die Implementierung der \textit{Support Vector Machine} wurde das Open-Source-Framework \texttt{scikit-learn} 0.24 \citep{JMLR:v12:pedregosa11a} verwendet. Neben den mathematischen Parametern, die die gesuchte Hyperfläche beschreiben und sich durch den Lernprozess iterativ geeigneten Werten annähern, existieren eine Reihe an sogenannten \textit{Hyperparametern}, die von dem Entwickler manuell festgelegt werden. Darunter fallen beispielsweise die Wahl von $\epsilon$ und C\footnote{Bezeichnung für den sogenannten Regularisierungs-Parameter. Ein hoher Grad an Regularisierung führt dazu, ein einfacheres Modell zu bevorzugen, das weniger von den konkreten Eingabedaten abhängt und stattdessen versucht, allgemeine Muster zu erkennen. So soll overfitting verhindert werden.}, sowie diverse Parameter bei der Vorverarbeitung der Eingabetexte. Das \texttt{scikit-learn}-Framework stellt für die Suche nach geeigneten Hyperparametern die Funktion \texttt{GridSearchCV} bereit. Diese durchläuft automatisch den gesamten Hyperparameter-Raum, indem automatisch für jede Kombination ein Modell trainiert und die Ergebnisse sortiert und gespeichert werden. 

\paragraph{Kreuzvalidierung} Zur Messung der Vorhersagegenauigkeit der trainierten Modelle kam die 5-fache Kreuzvalidierung zum Einsatz. Die Eingabedaten werden hierbei in fünf gleich große, disjunkte Teilmengen unterteilt. Das Modell wird zunächst auf eine Trainingsmenge, bestehend aus vier der fünf Teilmengen, trainiert. Anhand der noch ungesehenen Datenpaare aus der fünften Menge wird das Modell getestet. Als Verlustfunktion kommt \textit{mean absolute error} zum Einsatz, d.h. die mittlere Abweichung der Vorhersage zu dem tatsächlichen Wert. Im Gegensatz zu anderen, ebenfalls gebräuchlichen Metriken (z.B. $R^2$ oder mean squared error) erlaubt die mittlere Abweichung eine intuitive Einschätzung der Vorhersagegenauigkeit eines Modells. Eine Abweichung von 1 entspricht genau einem Punkt auf der betrachteten Score-Skala.

Anschließend wird der Prozess weitere vier Male wiederholt, wobei beim Training des Modells jedes mal eine andere Menge zu Testzwecken ausgelassen wird. Der Durchschnittswert der fünf Durchläufe wird zuletzt als Wert für die gewählte Hyperparameter-Kombination gespeichert. Die Anwendung des beschriebenenen Kreuzvalidierungsverfahren reduziert den Einfluss zufällig ausgewählter Trainings- und Testdaten.

Die folgenden Parameterkombinationen wurden untersucht:

\begin{table}[htb]
    \begin{tabular}{|l|l|l|}
    \hline
    Stufe          & Parameter    & mögliche Werte                                                \\ \hline
    word embedding & preprocessor & \makecell{lower, lower+clean, lower+clean+stem, \\ lower+clean+stem+rmstop} \\ \hline
    word embedding & tokenizer     & word, char, char\_wb                                          \\ \hline
    word embedding & ngram range\footnote{Die \textit{ngram range} gibt die minimale und maximale Länge der generierten N-Gramme an.}  & $\{(n, n+m), n \in \{1, 2, 4, 6, 8\}, m \in {0, 2, 4, 6, 8, 10}\}$                            \\ \hline
    SVR            & kernel       & \makecell{linear, polynomiell, radiale Basisfunktion,\\sigmoidal} \\ \hline
    SVR            & $\epsilon$      & 0.05, 0.1, 0.15                                               \\ \hline
    SVR            & C            & $\{10^{-n} | n \in \mathbb{Z} \cap [-1, 3] \} \cup \{5, 15\}$            \\ \hline
    \end{tabular}%
\end{table}

Die höchste Performance wurde durch einen zeichenbasierten Tokenizer mit einer ngram range von 2 -- 12, die auch über Wortgrenzen hinaus geht, erreicht. Es fließen also alle Zeichenketten mit einer Länge zwischen 2 und 12 Zeichen in die Eingabevektoren ein. Die Wahl von $\epsilon$ sowie das Filtern von Stoppwörtern hatte keinen statistisch signifikanten Einfluss auf das Ergebnis. Für den Regularisierungsparameter C erreichte ein Wert von 1 das beste Ergebnis (siehe Abbildung \ref{fig:mae_c}).

\begin{figure}[htb]
    \captionsetup{justification=centering}
    \centering
    \includegraphics[width=0.6\textwidth]{mae_C.png}
    \caption{Geringste und mittlere Abweichung nach Wahl des Regularisierungsparameters C}
    \label{fig:mae_c}
\end{figure}

\subsection{Ergebnisse}
Nach der Ermittlung der besten Hyperparameter wurde für die Visitentexte \texttt{Visite\_ZNS} und \texttt{Visite\_Pflege} jeweils ein Modell zur Vorhersage der Scores RASS und GCS entwickelt. Jedes dieser vier Modelle wurde mit einer sukzessive steigenden Anzahl an Trainingspaaren trainiert, um den Zusammenhang zwischen der Anzahl der Eingabedaten sowie der Leistung des Modells bei noch unbekannten Daten zu ermitteln (siehe Abbildung \ref{fig:svm_perf}). Für jeden der Trainingsvorgänge wurde eine Stichprobe der Größe n aus der Menge der entsprechenden Text-Wert-Paare betrachtet, deren Abstand zueinander 45 Minuten oder weniger betrug (siehe Abschnitt \ref{sec:pairgen}). 

\begin{figure}[htbp]
    \centering
    \subfloat[Vorhersage von RASS]{\includegraphics[width=1\textwidth]{alt_mae_rass}} \\
    \subfloat[Vorhersage von GCS]{\includegraphics[width=1\textwidth]{alt_mae_gcs}} 
    \caption{}
    \label{fig:svm_perf}
\end{figure}

Die Wahl von 45 Minuten als maximaler Abstand zwischen den Eintragungen ermöglich\-te für alle vier Kombinationen, bis zu 9000 Paare zum Trainieren des Modells zu finden, deren zeitliche Nähe einen starken Zusammenhang zwischen Ein- und Ausgabewert vermuten lässt. Um eine gute Vergleichbarkeit zwischen den verschiedenen Durchläufen zu ermöglichen, wurde für alle Stichproben die Menge der Paare mit Abstand $\leq$ 45 Minuten betrachtet -- auch wenn eine sehr geringe Stichprobengröße einen niedrigeren maximalen Abstand, und somit eine höhere Korrelation zwischen Ein- und Ausgabewert, ermöglicht hätte.

Unabhängig von der Anzahl der Trainingspaare wurde anschließend jedes Modell auf einem Testdatensatz der Größe n=1000 getestet. Bei der Wahl dieser Testdaten wurde darauf geachtet, nur solche Paare zu verwenden, die noch nicht beim Trainieren des Modells verwendet wurden. Eine Überschneidung hätte einen unfairen Vorteil für das Modell zur Folge und würde seine Leistung bei der Vorhersage von unbekannten Daten nicht korrekt wiederspiegeln.

Als Maßstab zur Bewertung der Modelle wurde erneut die mittlere absolute Abweichung der Vorhersagen zu den tatsächlichen Werten bestimmt. Der beschriebene Trainings- und Testprozess wurde für jedes Modell und jede Stichprobengröße zehn mal wiederholt, um Ungleichmäßigkeiten in der Qualität der zufällig ausgewählten Trainingsdaten entgegenzuwirken.


Abbildung \ref{fig:svm_perf} zeigt deutlich, dass unabhängig von der Art der Ein- und Ausgabedaten die Vorhersagegenauigkeit der Modelle steigt, je mehr Wertepaare zum Trainieren genutzt werden. Dieser Effekt ist bei einer geringen Anzahl an Trainingspaaren am stärksten ausgeprägt und lässt mit steigender Anzahl der Paare nach. Während sich die Abweichung der Vorhersagen einer unteren Schranke anzunähern scheint, steigt die benötigte Zeit für das Trainieren des Modells jedoch exponentiell (siehe Abbildung \ref{fig:fittime}). In der Praxis gilt es also, eine Abwägung zwischen der verfügbaren Rechenleistung und der benötigten Vorhersagegenauigkeit des Modells zu treffen.
 
\begin{figure}[htb]
    \captionsetup{justification=centering}
    \centering
    \includegraphics[width=1\textwidth]{fit_time.png}
    \caption{Mittlere Abweichung bei der Vorhersage von GCS und RASS in Zusammehang mit der benötigten Zeit zum Trainieren des Modells.}
    \label{fig:fittime}
\end{figure}

\subsubsection{Verteilung der ausgegebenen Werte}
Um zu überprüfen, ob das Modell einige Score-Werte besonders häufig vorhersagt, wurden erneut zwei Modelle anhand von jeweils 9000 Trainingspaaren trainiert und Vorhersagen für 1000 Eingabetexte getroffen. Hierbei wurden Texte der Kategorie \texttt{Visite\_ZNS} verwendet, welche für RASS und GCS genauere Ergebnisse ermöglicht als \texttt{Visite\_Pflege} (siehe Abbildung \ref{fig:svm_perf}). Beim Vergleich der vorhergesagten mit den tatsächlichen Werten fällt auf, dass das Modell sowohl für RASS als auch GCS "konservative" Schätzungen nahe der Mitte des Spektrums möglicher Werte höher priorisiert (siehe Abbildung \ref{fig:svm_hists}).

Die vorhergesagten Werte für RASS (türkise Balken) folgen im Allgemeinen der Verteilung der tatsächlichen Werte der Stichprobe (rot). Allerdings wurde für keinen der Texte ein RASS-Wert von 3 oder 4 bestimmt, obwohl diese Werte in der Original-Stichprobe 21 und 2 mal vorkamen. Dafür wurde ein RASS-Wert von 0 bei 421 Texten vorhergesagt, während dieser Wert in den Ausgangsdaten nur 342 mal vertreten war. Die Verteilung bei der Vorhersage der GCS-Werte folgt einem ähnlichen Muster. Die Grenzwerte 3 respektive 15 treten hier in der Original-Stichprobe besonders häufig auf, werden von dem Modell aber deutlich seltener ausgegeben. Die Häufigkeitsverteilung der anderen Vorhersagen folgt dafür annähernd der Verteilung der Originalwerte.

\begin{figure}[ht]
    \centering
    \subfloat[Vorhersage von RASS]{\includegraphics[width=.48\textwidth]{rass_hist}}
    \subfloat[Vorhersage von GCS]{\includegraphics[width=.48\textwidth]{gcs_hist}} 
    \caption{}
    \label{fig:svm_hists}
\end{figure}


\subsubsection{Filtern von RASS-Vorkommnissen}

35 \% ($n=8426$) der bereitgestellten Texte aus Kategorie \texttt{Visite\_ZNS} enthalten die Zeichenkette \texttt{'RASS=n'} oder eine Variation davon. Dabei ist zu beachten, dass sich selbst diese Angaben häufig von Eintragungen des RASS-Wertes in enger zeitlichen Nähe unterscheiden (siehe Abschnitt \ref{section:genauigkeit_der_daten}).

Um zu überprüfen, in welchem Umfang dieser Inhalt von dem Modell erkannt wird und in die Vorhersage des RASS-Wertes einfließt, wurden erneut je fünf Modelle mit 1000, 2000 und 3000 Trainingspaaren der Kombination \texttt{Visite\_ZNS:RASS} entwickelt. Es wurden nur solche Trainingspaare verwendet, die die o.g. Zeichenkette enthalten. Für jede Stichprobe wurde zunächst ein reguläres Modell trainiert, sowie danach eine Variation, bei der Inhalte der Form \texttt{'RASS=n'} vor der Tokenisierung hinausgefiltert wurden.

\begin{figure}[htb]
    \captionsetup{justification=centering}
    \centering
    \includegraphics[width=0.6\textwidth]{filter_rass.png}
    \caption{}
    \label{fig:filterrass}
\end{figure}

Abbildung \ref{fig:filterrass} zeigt, dass die Modelle, bei denen die Angabe eines expliziten RASS-Wertes im Voraus entfernt wurde, deutlich schlechter abschnitten als diejenige, denen diese Information zur Verfügung stand. Dies ist ein Indiz für die Fähigkeit des Modells, ohne explizites Wissen über die Bedeutung des Textes wichtige Informationen zu identifizieren und in die Berechnung des endgültigen Score-Wertes einfließen zu lassen.
	
	\chapter{Extreme Learning Machine}\label{chapter:ELM}
	\section{Vorverarbeitung der Eingabedaten}
Rechtschreibprüfung: wie viele wörter in den jeweiligen texten, wie viele unique? welches sind die häufigsten? Bla

% bei spell checker eher für einen naiven ansatz entschieden:
% wörter, die öfter als (3) mal vorkommen werden schon als
% richtig angesehen, auch wenn da viele falsch geschriebene
% wörter mit dabei sind. würde man aber nur wörter ansehen
% die noch öfters vorkommen, würde man auch wörter korriegieren
% die eigentlich richtig sind, nur weil sie nicht oft genug
% vorkamen, um als richtig angesehen zu werden:
% 'spannung' > spaltung
% 'verformung' > versorgung
Nur offensichtliche tippfehler korrigiert.
Beschreiben: Verbesserung der featurization der eingabedaten durch word2vec. verbesserung auch durch spell correction, weil es viele falsch geschriebene wörter gibt, die dann in word2vec gar nicht auftauchen. (zeigen, wie sich die anzahl der unbekannten wörter durch spell correction verringert!)

Vergleich unbekannte/bekannte wörter mit spell correction, substitutions etc und ohne.

\section{Vektorisierung mittels word2vec}
hier: wie funktioniert w2v, was ist performance von svm aus abschnitt 3.1 mit word2vec?

\section{Netzwerk}
was ist eine ELM \citep{huangExtremeLearningMachine2006}, was für ne toolbox hab ich verwendet \citep{akusokHighPerformanceExtremeLearning2015}

\section{Ergebnisse}
Das modell wurde anhand des in Abschnitt 3.2.1 beschriebenen kreuzvalidierungsverfahren getestet. bla.
	
	\chapter{Fazit}
	Im Rahmen der vorliegenden Arbeit wurde die Konzeption, Entwicklung und Bewertung eines Machine Learning-Modells zur Vorhersage medizinischer Score-Werte auf Basis von Visitentexten behandelt. Die Anwendung der Support Vector Regression wurde am Beispiel der Scores RASS und GCS sowie der Eingabetexte \texttt{Visite\_ZNS} und \texttt{Visite\_Pflege} vorgestellt. Das gleiche Konzept lässt sich mit nur kleinen Anpassungen auf andere Scores oder Eingabetexte übertragen.

Unter Verwendung der besten Hyperparameter \refsec{sec:hyperparams} und jeweils 9000 Trainingspaaren wurden folgende Ergebnisse bei der Vorhersage der Score-Werte ermittelt:

\begin{table}[h]
    \centering
    \begin{tabular}{rcc}
        Eingabetext
        & Score
        & MAE\\
        \midrule
        \texttt{Visite\_ZNS}    & RASS & 0,979 \\
        \texttt{Visite\_Pflege} & RASS & 1,09 \\
        \texttt{Visite\_ZNS}    & GCS  & 1,66 \\
        \texttt{Visite\_Pflege} & GCS  & 2,12 \\
        \bottomrule
    \end{tabular}
    \caption{mittlere absolute Abweichung der Vorhersagen}
\end{table}

Zukünftig können die Ausgaben des Modells beispielsweise genutzt werden, um sie mit den tatsächlich eingetragenen Werten zu vergleichen. Derartige Untersuchungen könnten Rückschlüsse über die bisherige Qualität der Datenerfassung auf den Intensivstationen der Charité ermöglichen und dazu beitragen, Vorgänge der Datenerfassung zu optimieren. Weiterhin kann analysiert werden, bei welchen Texten besonders große Abweichungen zwischen eingetragenem Wert und Vorhersage des Modells auftreten. Derartige Abweichungen könnten ein Indiz dafür sein, zu welchen Zeiten die Qualität der Datenerfassung ein besonders hohes Potenzial zur Verbesserung bietet.

Der vollständige Quelltext zur Datenaufbereitung, dem Machine Learning-Modell sowie zu der Arbeit selbst ist frei unter GitHub verfügbar: \url{https://github.com/mahathu/bachelorarbeit}. Insbesondere bietet das Python-Skript \texttt{predict.py} im Verzeichnis \texttt{models/SVM/} die Möglichkeit, anhand vortrainierter Modelle mit je 12500 Trainingspaaren GCS- und RASS-Werte von beliebigen Eingabetexten zu bestimmen.

	% \begin{appendices}
	% 	\chapter{Stoppwörter}
	% 	\begin{verbatim}
	% 		>>> from nltk.corpus import stopwords
	% 		>>> len(stopwords.words('german'))
	% 		232
	% 		>>> stopwords.words('german')		
	% 	\end{verbatim}
	% \end{appendices}

	\newpage
	\bibliography{literatur}

	% Selbstständigkeitserklärung (ist auf Seite 3 von deckblatt.pdf)
	\includepdf[pages={3}]{Deckblatt/deckblatt.pdf}
\end{document}