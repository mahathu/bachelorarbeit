\documentclass[
	a4paper,
	pagesize,
	pdftex,
	12pt,
	twoside, % + BCOR darunter: für doppelseitigen Druck aktivieren, sonst beide deaktivieren
	BCOR=5mm, % Dicke der Bindung berücksichtigen (Copyshop fragen, wie viel das ist)
	ngerman,
	fleqn,
	final,
	]{scrartcl}
\usepackage{ucs}
\usepackage[utf8x]{inputenc} % Eingabekodierung: UTF-8
\usepackage{fixltx2e} % Schickere Ausgabe
\usepackage[T1]{fontenc} % ordentliche Trennung
\usepackage[ngerman]{babel}
\usepackage{lmodern} % ordentliche Schriften
\usepackage[unicode=true]{hyperref}
\usepackage{setspace,graphicx,tikz,tabularx} % für Elemente der Titelseite
\usepackage[draft=false,babel,tracking=true,kerning=true,spacing=true]{microtype} % optischer Randausgleich etc.

\begin{document}

\input{Institutsvorlage}
% \titel{Analyse von Visitentexten mittels maschinellem Lernen zur Optimierung der Leitlinienadhärenz auf einer Intensivstation} % Titel der Arbeit
\titel{Vorhersage von intensivmedizinischen Scores auf Basis von Visitentexten mittels maschinellem Lernen}
\typ{Bachelorarbeit} % Typ der Arbeit:  Diplomarbeit, Masterarbeit, Bachelorarbeit
\grad{Bachelor of Science (B. Sc.)} % erreichter Akademischer Grad
\autor{Martin Hoffmann} % Autor der Arbeit, mit Vor- und Nachname
\gebdatum{06.09.1998} % Geburtsdatum des Autors
\gebort{Berlin} % Geburtsort des Autors
\gutachter{Prof. Dr. med. Dr. rer. nat. Felix Balzer}{Dr. med. Fridtjof Schiefenhövel} % Erst- und Zweitgutachter der Arbeit
\mitverteidigung % entfernen, falls keine Verteidigung erfolgt
\makeTitel

% Erzeugen der Selbständigkeitserklärung auf einem neuen Blatt:
\selbstaendigkeitserklaerung{24. Juli 2020}

\end{document}
